\section{Proof of some Lemmas}\label{sec:CA.proofs} 
% We choose the boundary version of the classical \texttt{Dofi-Dofi} definition of the stabilization term (see \cite[Section 4.2]{brenner.sung:2018:virtual} and \cite{beirao-da-veiga.brezzi.ea:2013:basic}),
% \begin{align}\label{def:bd.stab}
%     S^D(w_h, v_h) = \sum_{i=1}^{N_{\partial D}^{\texttt{dof}}} \chi_i(w_h) \; \chi_i(v_h) \quad \forall w_h, v_h \in V_h(D),
% \end{align}
% where operator $\chi_i(\cdot)$ associates the function with its $i$-th degree of freedom, and $N_{\partial D}^{\texttt{dof}}$ is the number of boundary degrees of freedom associated with the element in $D \in \Th$. We have the following property associated with this choice of stabilization (see Remark 4.3 \cite{brenner.sung:2018:virtual}),
% \begin{align}
%     % \sum_{i=1}^{N_{\partial D}^{\texttt{dof}}} \chi_i(q)^2 \approx \norm{L^\infty(\partial D)}{q}^2 \quad \forall q \in \Poly_k(\partial D), \label{eq:stab.p1}\\
%     \sum_{i=1}^{N_{\partial D}^{\texttt{dof}}} \chi_i(w)^2 \lesssim \norm{L^\infty(D)}{w}^2 \quad \forall w \in C(\bar{D}) \label{eq:stab.p2},
% \end{align}
% where the hidden constants depend on $\Theta$ and $k$.

\subsection{Proof of Lemma \ref{lem:est.mesh.dependent}}\label{sec:pf:lem:est.mesh.dependent}
Observing that $\Pi^1_{k,h}((I - \Pi^1_{k,h})\Ih v) = 0$ due to linearity and idempotency of $\Pi^0_{k,h}$ we deduce that
\begin{align*}
    \norm{h_0}{\Ih{v} - \Pi^1_{k,h} \Ih{v}}^2 &= \sum_{D \in \Th} S^D((I - \Pi^1_{k,h}) \Ih v, (I - \Pi^1_{k,h}) \Ih v) \\
    &= \sum_{D \in \Th} \sum_{i=1}^{N_{\partial D}^{\texttt{dof}}} \chi_i((I - \Pi^1_{k,h}) \Ih v)^2 \\
    &\hspace{-0.15cm}\overset{\eqref{eq:stab.p2}}{\lesssim} \sum_{D \in \Th} \norm{L^\infty(D)}{(I - \Pi^1_{k,h}) \Ih v}^2 \\
    &\overset{\eqref{eq:Sobolev.inequality}}{\lesssim} \sum_{D \in \Th} h_D^{-2} \norm{L^2(D)}{(I - \Pi^1_{k,D}) \Ih v}^2 + \seminorm{H^1(D)}{(I - \Pi^1_{k,D}) \Ih v}^2 \\
    &\qquad+ h_D^{2\delta} \seminorm{H^{1+\delta}(D)}{(I - \Pi^1_{k,D}) \Ih v}^2 
\end{align*}
Now splitting $(I-\Pi^1_{k,D}) \Ih{v} = \Ih{v} - v + v - \Pi^1_{k,D} \Ih{v}$ and using the approximation properties of $\Ih$ from \eqref{eq:ID.approximation} and \eqref{eq:ID.approximation.L2}, we get the bound on the first term stated in \eqref{eq:est.Ih.h.norm}. Similarly, the second term in \eqref{eq:est.Ih.h.norm} follows from the definition of $\seminorm{1,h}{\cdot}$, the above mentioned split followed by a triangle inequality and the approximation properties of the interpolation operator stated in Lemma \ref{lem:ID.properties}.

\smallskip

Using the linearity and idempotency of $\Pi^1_{k,h}$ and \eqref{eq:stab.p2}, we can get the following
\begin{align*}
    \norm{h_\beta}{v- \Pi^1_{k,h}{v}}^2 &\lesssim \sum_{D \in \Th} \norm{L^\infty(D)}{v - \Pi^1_{k,D} v}^2 + \beta \norm{L^2(D)}{v - \Pi^1_{k,D} v}^2 \\
    &\hspace{-0.5cm}\overset{\eqref{eq:stab.p2},\eqref{eq:Sobolev.inequality}}{\lesssim} \sum_{D \in \Th} h_D^{-2} \norm{L^2(D)}{(I - \Pi^1_{k,D}) v}^2 + \seminorm{H^1(D)}{(I - \Pi^1_{k,D}) v}^2 \\
    &\quad\qquad+ h_D^{2\delta} \seminorm{H^{1+\delta}(D)}{(I - \Pi^1_{k,D}) v}^2 + \beta \norm{L^2(D)}{v - \Pi^1_{k,D} v}^2.
\end{align*}
Now using the approximation property of $\Pi^1_{k,D}$ in Lemma \ref{lem:H1p.properties} leads to the desired estimate on the first term in \eqref{eq:est.hbeta.norm}. Similarly, we have
\begin{align*}
    \norm{h_\beta}{v - \Ih{v}}^2 &= \sum_{D \in \Th} \seminorm{H^1(D)}{\Pi^1_{k,D}(v - \ID v)}^2 + \sum_{i=1}^{N_{\partial D}^{\texttt{dof}}} \chi_i((I-\Pi^1_{k,D})(I-\ID{})v)^2
    + \beta \norm{L^2(D)}{\Pi^0_{k,D}(v - \ID v)}^2 \\
    &= \sum_{D \in \Th} \seminorm{H^1(D)}{\Pi^1_{k,D}(v - \ID v)}^2 + \sum_{i=1}^{N_{\partial D}^{\texttt{dof}}} \chi_i(\Pi^1_{k,D}(I-\ID{})v)^2
    + \beta \norm{L^2(D)}{\Pi^0_{k,D}(v - \ID v)}^2 \\
    &\lesssim \sum_{D \in \Th} \seminorm{H^1(D)}{v - \ID v}^2 +  \norm{L^\infty(D)}{\Pi^1_{k,D}(I-\Pi^1_{k,D})v}^2 
    + \beta \norm{L^2(D)}{v - \ID v}^2,
\end{align*}    
where in the passage to the second equality we use the fact that $\chi_i(v - \ID{v}) = 0$ for boundary degrees of freedom by definition of interpolation operator \eqref{eq:ID.boundary}. In the last inequality we used the stability of projectors \eqref{eq:stability.H1p}, \eqref{eq:stability.L2p} and the property \eqref{eq:stab.p2}. The first and last terms are bounded using the approximation properties of $\ID$ from Lemma \ref{lem:ID.properties}. The second term can be estimated as in \cite[(4.24)]{brenner.sung:2018:virtual}. Combining all the resulting bounds leads to the desired estimate on the second term in \eqref{eq:est.hbeta.norm}.


























% \subsection{Proof of Lemma \ref{lem:Jh.approx}}\label{sec:pf:lem.Jh.approx}
%  We split the defect $ J_h v_h - v_h$ using $\Pi^0_{k,h}$ as 
%     \begin{align}
%         \norm{L^2(\Omega)}{J_h v_h - v_h} \leq \norm{L^2(\Omega)}{J_h v_h - \Pi^0_{k,h} v_h} + \norm{L^2(\Omega)}{\Pi^0_{k,h} v_h - v_h}. \label{eq:Jh.approximation.1}
%     \end{align}
%     The first term in \eqref{eq:Jh.approximation.1} is bounded using the definition of $J_h$ in \eqref{def:Jh} as follows:
%     \begin{align*}
%         \norm{L^2(\Omega)}{\Pi^0_{k,h} v_h - J_h v_h}^2 &= (\Pi^0_{k,h} v_h - J_h v_h,\Pi^0_{k,h} v_h - J_h v_h) \\
%         &= (\Pi^0_{k,h} v_h - J_h v_h, \Pi^0_{k,h} v_h) \\
%         &= (\Pi^0_{k,h} v_h  - J_h v_h, \Pi^0_{k,h} v_h - v_h).
%     \end{align*}
%     Thus, upon using Causchy-Schwarz inequality and \eqref{eq:approxL2.L2p}, we have
%     $$ \norm{L^2(\Omega)}{\Pi^0_{k,h} v_h - J_h v_h} \lesssim \left(\sum_{D \in \Th} \norm{L^2(D)}{ \Pi^0_{k,D} v_h - v_h}^2\right)^{1/2} \lesssim \left( \sum_{D \in \Th} h_D^2 \seminorm{H^1(D)}{v_h}^2 \right)^{1/2}.$$


% \subsection{Proof of Lemma \ref{lem:Jh.stability}}\label{sec:pf:lem.Jh.stability}
%    Testing with $v_h = J_h w$ and using Cauchy-Schwarz yields the stability estimate for $J_h$,
%     $$\norm{L^2(\Omega)}{J_h w} \leq \norm{L^2(\Omega)}{\Pi^0_{k,h} w} \overset{\eqref{eq:stability.L2p}}{\lesssim} \norm{L^2(\Omega)}{w}.$$

%     To prove the $H^1$-stability of $J_h$, we proceed as follows,
%     \begin{align}
%         \seminorm{H^1(\Omega)}{J_h v_h}^2 &\lesssim \seminorm{H^1(\Omega)}{J_h v_h - v_h}^2 + \seminorm{H^1(\Omega)}{v_h}^2
%         = \sum_{D \in \Th} \seminorm{H^1(D)}{J_h v_h - v_h}^2 + \seminorm{H^1(\Omega)}{v_h}^2. \label{eq:Jh.approxH1.1}
%     \end{align}
%     Using the inverse estimate in Lemma \ref{lem:inverse.inequality} for $\delta_h := J_h v_h - v_h \in V_h(D)$, we have
%     \begin{align}
%         \seminorm{H^1(D)}{\delta_h}^2 &\lesssim h_D^{-2} \tnorm{k,D}{\delta_h}^2 \nonumber\\
%         &= h_D^{-1} \sum_{e \in \ED} \norm{L^2(e)}{\Pi^0_{k,e} \delta_h}^2 + h_D^{-2}\norm{L^2(D)}{\Pi^0_{k-2,D} \delta_h}^2 \nonumber \\
%         &\leq h_D^{-1} \norm{L^2(\partial D)}{\delta_h}^2 + h_D^{-2}\norm{L^2(D)}{\delta_h}^2, \label{eq:Jh.approxH1.2}
%     \end{align}
%     where we used the stability of $\Pi^0_{k,e}$ and $\Pi^0_{k-2,D}$ from \eqref{eq:stability.L2p}. Now using a multiplicative trace inequality \cite[Theorem 1.6.6]{Brenner.Scott:2008:MTFEM} followed by Young's inequality for $\epsilon > 0$, yields
%     \begin{align*}
%         h_D^{-1} \norm{L^2(\partial D)}{\delta_h}^2 \lesssim h_D^{-1} \norm{L^2(D)}{\delta_h} \norm{H^1(D)}{\delta_h}
%         \leq \frac{1}{2\epsilon} h_D^{-2} \norm{L^2(D)}{\delta_h}^2 + \frac{\epsilon}{2} \norm{H^1(D)}{\delta_h}^2.
%     \end{align*}
%     Substituting the above in \eqref{eq:Jh.approxH1.2}, gives
%     \begin{align*}
%         \seminorm{H^1(D)}{\delta_h}^2 \lesssim \frac{1}{2\epsilon} h_D^{-2} \norm{L^2(D)}{\delta_h}^2 + \frac{\epsilon}{2} \norm{H^1(D)}{\delta_h}^2.
%     \end{align*}
%     Hence, applying the approximation property of $J_h$ from Lemma \ref{lem:Jh.approx} {\color{blue} alongwith quasi-uniformity of the mesh}, and Poincar\'e-Friedrichs inequality yields    
%     \begin{align*}
%         \seminorm{H^1(\Omega)}{\delta_h}^2 = \sum_{D \in \Th} \seminorm{H^1(D)}{\delta_h}^2 &\lesssim \left(1+\frac{1}{2\epsilon}\right) \sum_{D \in \Th} h_D^{-2} \norm{L^2(D)}{\delta_h}^2 + \frac{\epsilon}{2} \sum_{D \in \Th} \norm{H^1(D)}{\delta_h}^2 \nonumber\\
%         &\approx \left(1+\frac{1}{2\epsilon}\right) h^{-2} \sum_{D \in \Th}  \norm{L^2(D)}{\delta_h}^2 + \frac{\epsilon}{2} \sum_{D \in \Th} \norm{H^1(D)}{\delta_h}^2 \nonumber\\
%         &\lesssim \left(1+\frac{1}{2\epsilon}\right) \sum_{D \in \Th} \seminorm{H^1(D)}{v_h}^2 + \frac{\epsilon}{2} \norm{H^1(\Omega)}{\delta_h}^2 \nonumber\\
%         &\lesssim\left(1+\frac{1}{2\epsilon}\right) \seminorm{H^1(\Omega)}{v_h}^2 + \frac{\epsilon}{2} \seminorm{H^1(\Omega)}{\delta_h}^2.
%     \end{align*}
%     Choose $\epsilon > 0$ sufficiently small to absorb the last term on the right-hand side into the left-hand side, and substitute the resulting bound in \eqref{eq:Jh.approxH1.1} to complete the proof. 