\section{Convergence analysis}\label{sec:CA} 

We start by collecting some mathematical tools which will be helpful in the forthcoming analysis. 

\begin{lemma}[Sobolev inequality]\label{lem:Sobolev.inequality} Given any $\delta > 0$,
    $$\norm{L^\infty(D)}{v} \lesssim h_D^{-1} \norm{L^2(D)}{v} + \seminorm{H^1(D)}{v} + h_D^{\delta} \seminorm{H^{1+\delta}(D)}{v} \quad \forall v \in H^{1+\delta}(D).$$
\end{lemma}

\begin{lemma}[Bramble-Hilbert estimates]
    Under the mesh regularity Assumption \ref{meshreg_assum.1}, given any $\delta > 0$, there exists a positive constant independent of $h_D$ such that 
    $$\inf_{p \in \Poly_k(D)} \seminorm{H^1(D)}{\lambda - \psi} \lesssim h_D^{\min(\delta,k)} \seminorm{H^{1+\delta}(D)}{\lambda}, \quad \forall \lambda \in H^{1+\delta}(D).$$
\end{lemma}
% \begin{proof}
%     Using arguments detailed in \cite[Lemma 4.3.8]{Brenner.Scott:2008:MTFEM} and then using interpolation theory \cite[Chapter 14]{Brenner.Scott:2008:MTFEM} to extend to fractional order Sobolev spaces.
% \end{proof}
\begin{lemma}[Trace inequality]\label{lem:trace.inequality}
    Let $e$ be an edge of $D \subset \R^2$. Then, for all $v \in H^{1+\delta}(D)$ with given $\delta > 0$, we have
    $$h_D^{2 \delta} \seminorm{H^{1/2+\delta}(e)}{v}^2 \lesssim \seminorm{H^1(D)}{v}^2 + h_D^{2\delta} \seminorm{H^{1+\delta}(D)}{v}^2.$$
\end{lemma}

\begin{lemma}[$H^1$-projector stability and approximation]\label{lem:L2p.properites} Given $\delta > 0$,
    \begin{align}
        \seminorm{H^1(D)}{\Pi^1_{k,D} v} &\leq \seminorm{H^1(D)}{v} \quad \forall v \in H^1(D), \label{eq:stability.H1p} \\
        \seminorm{H^1(D)}{v - \Pi^1_{k,D} v} &\lesssim h_D^{\min(\delta,k)} \seminorm{H^{1+\delta}(D)}{v} \quad \forall v \in H^{1+\delta}(D). \label{eq:approxH1.H1p} 
        % \norm{L^2(D)}{v - \Pi^1_{k,D} v} &\lesssim h_D^{l+1} \seminorm{H^{l+1}(D)}{v} \quad \forall v \in H^{l+1}(D), \quad 0 \leq l \leq k. \label{eq:approxL2.H1p} \\
        % \seminorm{H^1(D)}{v - \Pi^1_{k,D} v} &\lesssim h_D^{l} \seminorm{H^{l+1}(D)}{v} \quad \forall v \in H^{l+1}(D), \quad 1 \leq l \leq k. \label{eq:approxH1.H1p}.
    \end{align}
\end{lemma}

\begin{lemma}[$L^2$-projector stability and approximation]\label{lem:H1p.properties}
    \begin{align}
       &\norm{L^2(D)}{\Pi^0_{k,D} v} \leq \norm{L^2(D)}{v} \quad \forall v \in L^2(D), \quad
        \norm{H^1(D)}{\Pi^0_{k,D} v} \leq \norm{H^1(D)}{v} \quad \forall v \in H^1(D), \label{eq:stability.L2p} \\
        &\norm{L^2(D)}{v - \Pi^0_{k,D} v} \lesssim h_D^{l+1} \seminorm{H^{l+1}(D)}{v} \quad \forall v \in H^{l+1}(D), \quad 0 \leq l \leq k, \label{eq:approxL2.L2p}
        % &\seminorm{H^1(D)}{v - \Pi^0_{k,D} v} \lesssim h_D^{l} \seminorm{H^{l+1}(D)}{v} \quad \forall v \in H^{l+1}(D), \quad 1 \leq l \leq k. \label{eq:approxH1.L2p} 
    \end{align}
\end{lemma}

The interpolation operator, which takes any sufficiently smooth function and maps it to the virtual element space. For $s > 1$, the global interpolation operator $\Ih{}: H^s(\Omega) \rightarrow V_h$ is the global counterpart of the local interpolation operator $\ID{}: H^s(D) \rightarrow V_h(D)$ for all $D \in \Th$ such that for any $v \in H^s(D)$,
\begin{align}
    \ID{v(p)} &= v(p) \quad \forall p \in \mathcal{N}^{\partial D}, \label{eq:ID.boundary} \\
    \Pi^0_{k-2,D} \ID{v} &= \Pi^0_{k-2,D} v \label{eq:ID.internal}. 
\end{align}
Here $\mathcal{N}^{\partial D}$ is the set of boundary degrees of freedom associated with the local virtual element space. 
% From \eqref{eq:ID.internal}, it follows that for $k \geq 2$, the interpolation $\ID{v}$ preserves the mean value of $v$ on each element $D$, i.e., given $v \in H^s(D) \cap L_0^2(\Omega)$,
% \begin{align}\label{rem:ID.meanvalue}
%     \int_D \ID{v} = \int_D \Pi^0_{k-2,D} (\ID{v}) = \int_D \Pi^0_{k-2,D} v = \int_D v = 0.
% \end{align}
% However, for $k = 1$, $\ID{v}$ is completely determined by boundary degrees of freedom \eqref{eq:ID.boundary} and does not necessarily preserve the mean value of $v$ on $D$.
    

\begin{lemma}[Interpolation operator stability and approximation]\label{lem:ID.properties}
    Given $\delta > 0$, we have
    \begin{align}
        \seminorm{H^1(D)}{\ID{v}} &\lesssim \seminorm{H^1(D)}{v} + h_D^{\delta} \seminorm{H^{1+ \delta}(D)}{v}, \quad \forall v \in H^{1+\delta}(D) \label{eq:ID.stability} \\
        \seminorm{H^1(D)}{v - \ID v} + \seminorm{H^1(D)}{\ID v - \Pi^1_{k,D} \ID v} &\lesssim h_D^{\min(\delta,k)} \seminorm{H^{1+\delta}(D)}{v}, \quad \forall v \in H^{1+\delta}(D)\label{eq:ID.approximation}
    \end{align}   
\end{lemma}

% We will also use the following inverse inequality for virtual element functions \cite[Lemma 2.19]{brenner.guan.ea:2017:some}.
% \begin{lemma}[Inverse inequality]\label{lem:inverse.inequality}
%     Under the mesh regularity assumptions \eqref{meshreg_assum.1}-\eqref{meshreg_assum.2} we have for all $v_h \in V_h(D)$,
%     \begin{align*}
%         \seminorm{H^1(D)}{v_h} \lesssim h_D^{-1} \tnorm{k,D}{v_h},
%     \end{align*}
%     where, $\tnorm{k,D}{\cdot}$ plays the role of $L^2$-norm and is defined as
%     $$\tnorm{k,D}{v_h}^2 = h_D \sum_{e \in \E_D} \norm{L^2(e)}{\Pi^0_{k,e} v_h}^2 + \norm{L^2(D)}{\Pi^0_{k-2,D} v_h}^2.$$
% \end{lemma}

In the subsequent analysis, we work with the following assumption:
\begin{assumption}\label{assum:coercivity}
    Let $\norm{h}{\cdot} = \sqrt{a_h(\cdot,\cdot)}$ denote the mesh-dependent energy norm such that 
    \begin{align}
        \seminorm{H^1(\Omega)}{v_h} &\lesssim \norm{h}{v_h} \quad \forall v_h \in V_h^0, \label{eq:coercivity.Dirichlet} \\
        \seminorm{H^1(\Omega)}{v_h} &\lesssim \norm{h}{v_h} \quad \forall v_h \in V_h \quad \mbox{with} \quad (v_h,1) = 0. \label{eq:coercivity.Neumann} 
    \end{align}
\end{assumption}

We also define the piecewise-broken $H^1$-seminorm by
\begin{align*}
    \seminorm{1,h}{v} := \left( \sum_{D \in \Th} \seminorm{H^1(D)}{v}^2 \right)^{\frac{1}{2}} \quad \forall v \in H^1(\Th).
\end{align*}

Some useful estimates in the mesh-dependent norm are in order.
\begin{lemma}\label{lem:est.mesh.dependent}
    The following estimates hold,
    \begin{align}
        \norm{h}{\Ih{v} - \Pi^1_{k,h} \Ih{v}} + \seminorm{1,h}{\Ih{v} - \Pi^1_{k,h} \Ih{v}} &\lesssim h^{\min(\delta,k)} \seminorm{H^{1+\delta}(\Omega)}{v} \quad \forall v \in H^{1+\delta}(\Omega), \label{eq:est.Ih.h.norm}
    \end{align}
\end{lemma}


% In order to prove optimal error estimates for $\xi_h$ without assuming extra regularity on $\rho$, we define the following global conforming linear operator,

% \begin{align}\label{def:Jh}
%     J_h: L^2(\Omega) \rightarrow V_h \quad \mbox{such that} \quad (J_h w, v_h) = (\Pi^0_{k,h} w, v_h) \quad \forall v_h \in V_h.
% \end{align}

% Since, $V_h \subset L^2(\Omega)$ is finite dimensional, the tuple $(V_h, (\cdot,\cdot))$ forms a Hilbert space. For any $w \in L^2(\Omega)$, the map $l_w: V_h \rightarrow \mathbb{R}$ defined by $l_w(v_h) := (\Pi^0_{k,h} w, v_h)$ is a bounded linear functional, with 
% $$ |l_w(v_h)| \leq \norm{L^2(\Omega)}{\Pi^0_{k,h} w} \norm{L^2(\Omega)}{v_h} = \sum_{D \in \Th} \norm{L^2(D)}{\Pi^0_{k,D} w} \norm{L^2(\Omega)}{v_h} \overset{\eqref{eq:stability.L2p}}{\lesssim} \norm{L^2(\Omega)}{w} \norm{L^2(\Omega)}{v_h}. $$ 
% Hence, by the Riesz representation theorem, there exists a unique $J_h w \in V_h$ such that $(J_h w, v_h) = l_w(v_h)$ for all $v_h \in V_h$. 

% \begin{lemma}[Approximation property of $J_h$]\label{lem:Jh.approx}
%     For all $v_h \in V_h$, we have
%     $$ \norm{L^2(\Omega)}{J_h v_h - v_h}^2 \lesssim \sum_{D \in \Th} h_D^2 \seminorm{H^1(D)}{v_h}^2.$$    
% \end{lemma}
% \begin{proof}
%    See Section \ref{sec:pf:lem.Jh.approx}.
% \end{proof}

% \begin{lemma}[Stability of $J_h$]\label{lem:Jh.stability}
%     The following hold,
%     \begin{align}
%         \norm{L^2(\Omega)}{J_h w} &\lesssim \norm{L^2(\Omega)}{w} \quad \forall w \in L^2(\Omega),  \label{eq:Jh.L2.stability} \\
%         \seminorm{H^1(\Omega)}{J_h v_h} &\lesssim \seminorm{H^1(\Omega)}{v_h} \quad \forall v_h \in V_h^0. \label{eq:Jh.H1.stability}
%     \end{align}
% \end{lemma}

% \begin{proof}
%     see Section \ref{sec:pf:lem.Jh.stability}.
% \end{proof}



\subsection{$\Omega$ is simply connected $(\gamma = 0)$} Our first aim is to estimate the error $\xi - \xi_h$. To this end, we first write down the error equation by testing \eqref{eq:P.rho} with $\psi_h \in V_h \subset H^1(\Omega)$ and subtracting it from \eqref{eq:P.rho},
\begin{align}\label{eq:erreq.rho}
    (\curl{\rho}, \curl{\psi_h}) - a_h(\rho_h,\psi_h) = (\bs{f}, \curl{(\psi_h - \Pi^1_{k,h}(\psi_h))}) \quad \forall \psi_h \in V_h.
\end{align}
We note that to obtain the above error equation we also used the definition of $\Pi^0_{k,h}$ and the fact that $\rho, \rho_h \in L_0^2(\Omega)$. Furthermore, in view of \eqref{eq:P.rho}, \eqref{eq:Ph.rhoh}, $\rho, \rho_h \in L_0^2(\Omega)$ and Assumption \ref{assum:coercivity}, we have the following relations:
\begin{align}\label{eq:curl.data}
    \norm{L^2(\Omega)}{\curl{\rho}} \leq \norm{L^2(\Omega)}{\bs{f}}, \quad \mbox{and} \quad \norm{L^2(\Omega)}{\curl{\rho_h}} \lesssim \norm{L^2(\Omega)}{\bs{f}}.
\end{align}

In the following lemma we estimate the error for $\rho_h$ in the dual norm using a duality argument.

\begin{lemma}\label{lem:rhoh.dual}
    For any $\epsilon > 0$, there exists a positive constant dependent on $\epsilon$ and independent of $h$ such that
    \begin{align}\label{eq:err.est.rho}
       | (\rho-\rho_h, \chi) | \lesssim h^{\min{((\pi / \omega) - \epsilon, k)}} \norm{H^1(\Omega)}{\chi} \norm{L^2(\Omega)}{\bs{f}} \quad \forall \chi \in H^1(\Omega).
    \end{align}

\end{lemma}

\begin{proof}
    Given arbitary $\chi \in H^1(\Omega)$, let $\lambda \in H^1(\Omega)$ solve the following dual problem:
\begin{align}\label{eq:dual.rho}
    (\curl{\psi}, \curl{\lambda}) + (\psi,1) (\lambda,1) = (\psi, \chi) \quad \forall \psi \in H^1(\Omega).
\end{align}
Testing \eqref{eq:dual.rho} with $\rho - \rho_h \in H^1(\Omega)$ and exploiting $\rho, \rho_h \in L_0^2(\Omega)$, we get
\begin{align}\label{eq:err.rho.1}
    (\rho - \rho_h,\chi) &= (\curl{(\rho - \rho_h)}, \curl{\lambda}) + (\rho - \rho_h,1) (\lambda,1) \nonumber\\
    &= (\curl{(\rho - \rho_h)}, \curl{(\lambda- \Ih{\lambda})} ) + (\curl{(\rho - \rho_h)}, \curl{\Ih{\lambda}}).
\end{align}
The first term in \eqref{eq:err.rho.1} is bounded using the Cauchy-Schwarz inequality and \eqref{eq:curl.data}, to get
\begin{align}\label{eq:err.rho.1.a}
    (\curl{(\rho - \rho_h)}, \curl{(\lambda- \Ih{\lambda})} ) &\leq \norm{L^2(\Omega)}{\curl{(\rho - \rho_h)}} \norm{L^2(\Omega)}{\curl{(\lambda - \Ih{\lambda})}}. \nonumber \\
    &\lesssim \norm{L^2(\Omega)}{\bs{f}} \seminorm{H^1(\Omega)}{\lambda - \Ih{\lambda}}.
\end{align}

The second term in \eqref{eq:err.rho.1} is bounded using the error equation \eqref{eq:erreq.rho} for $\psi_h = \Ih{\lambda}$ as follows:
\begin{align}\label{eq:err.rho.1.b}
    (\curl{(\rho - \rho_h)}, \curl{\Ih{\lambda}}) &= (\curl{\rho}, \curl{\Ih{\lambda}}) - (\curl{\rho_h}, \curl{\Ih{\lambda}}) \nonumber\\
    &= (\bs{f}, \curl{(\Ih{\lambda} - \Pi^1_{k,h} \Ih{\lambda})}) + a_h(\rho_h,\Ih{\lambda}) - (\curl{\rho_h}, \curl{\Ih{\lambda}}) \nonumber\\
    &\leq \norm{L^2(\Omega)}{\bs{f}} \seminorm{1,h}{(I-\Pi^1_{k,h})\Ih{\lambda}} + a_h(\rho_h,\Ih{\lambda}) - (\curl{\rho_h}, \curl{\Ih{\lambda}}).
\end{align}

The difference in \eqref{eq:err.rho.1.b} can be rewritten as follows:
\begin{align}
    &a_h(\rho_h,\Ih{\lambda}) - (\curl{\rho_h}, \curl{\Ih{\lambda}})  \nonumber\\
    &\quad=\sum_{D \in \Th} (\curl{\Pi^1_{k,D} \rho_h, \curl{\Pi^1_D {\ID{\lambda}}}}) + S^D((I - \Pi^1_{k,D})\rho_h, (I - \Pi^1_{k,D})\ID{\lambda}) -\nonumber\\
    &\qquad\qquad(\curl{\rho_h}, \curl{(\Ih{\lambda}-\Pi^1_{k,D}\ID{\lambda})}) - (\curl{\rho_h}, \curl{\Pi^1_{k,D}\ID{\lambda}}) \nonumber\\
    &\quad= \sum_{D \in \Th} S^D((I - \Pi^1_{k,D})\rho_h, (I - \Pi^1_{k,D})\ID{\lambda}) - (\curl{\rho_h}, \curl{(\ID{\lambda} - \Pi^1_{k,D}\ID{\lambda})}) \nonumber\\
    &\quad\leq  (\sum_{D \in \Th} S^D((I - \Pi^1_{k,D})\rho_h, (I - \Pi^1_{k,D})\rho_h))^{1/2} (\sum_{D \in \Th} S^D((I - \Pi^1_{k,D})\ID{\lambda}, (I - \Pi^1_{k,D})\ID{\lambda}))^{1/2} \nonumber\\
    &\qquad\qquad (\sum_{D \in \Th} \seminorm{H^1(D)}{\rho_h}^2)^{1/2} (\sum_{D \in \Th} \seminorm{H^1(D)}{\ID{\lambda} - \Pi^1_{k,D}\ID{\lambda}}^2)^{1/2} \nonumber\\
    &\quad\lesssim \norm{h}{(I - \Pi^1_{k,h})\rho_h} \norm{h}{(I-\Pi^1_{k,h})\Ih{\lambda}} + \seminorm{H^1(\Omega)}{\rho_h} \seminorm{1,h}{(I - \Pi^1_{k,h})\Ih{\lambda}}. \label{eq:err.rho.1.c}
\end{align}
Where in the passage to the second equality we used the definition of $\Pi^1_{k,D}$ projector followed by Cauchy-Schwarz inequality and finally the definition of $\norm{h}{\cdot}$ and $\seminorm{1,h}{\cdot}$. Using the linearity and idempotency of $\Pi^1_{k,D}$ and $\rho_h \in L_0^2(\Omega)$ we have that
\begin{align*}
    \norm{h}{(I - \Pi^1_{k,h} \rho_h)}^2 &= S((I - \Pi^1_{k,h})\rho_h, (I - \Pi^1_{k,h})\rho_h) \nonumber\\ 
    &\overset{\eqref{eq:Ph.rhoh}}{=} (\bs{f}, \curl{\Pi^1_{k,h} \rho_h}) - (\curl{\Pi^1_{k,h} \rho_h}, \curl{\Pi^1_{k,h} \rho_h}) \nonumber\\
    &\leq \sum_{D \in \Th} \norm{L^2(D)}{\bs{f}} \norm{H^1(D)}{\Pi^1_{k,D} \rho_h} + \norm{H^1(D)}{\Pi^1_{k,D} \rho_h}^2 \nonumber\\
    &\overset{\eqref{eq:stability.H1p}}{\leq}  \sum_{D \in \Th} \norm{L^2(D)}{\bs{f}} \norm{H^1(D)}{\rho_h} + \norm{H^1(D)}{\rho_h}^2 \overset{\eqref{eq:curl.data}}{\lesssim} \norm{L^2(\Omega)}{\bs{f}}^2.
\end{align*}
The regularity of the dual problem from Lemma \ref{lem:reg.pure-neumann}, the above bound, the estimate \eqref{eq:est.Ih.h.norm} with $\delta = (\pi/\omega)-\epsilon$, and \eqref{eq:curl.data} lead to the following bound on \eqref{eq:err.rho.1.c},
\begin{align*}
    a_h(\rho_h,\Ih{\lambda}) - (\curl{\rho_h}, \curl{\Ih{\lambda}}) \lesssim h^{\min{((\pi / \omega)-\epsilon,k)}} \norm{L^2(\Omega)}{\bs{f}} \seminorm{H^{1+(\pi/\omega)-\epsilon}(\Omega)}{\lambda}.  
\end{align*}

Substituting in \eqref{eq:err.rho.1.b}, again using \eqref{eq:est.Ih.h.norm} and the data dependence relation of the dual problem in Lemma \ref{lem:reg.pure-neumann} yield
\begin{align*}
    (\curl{(\rho - \rho_h)}, \curl{\Ih{\lambda}}) &\lesssim  h^{\min{((\pi / \omega)-\epsilon,k)}} \norm{L^2(\Omega)}{\bs{f}} \norm{H^{1}(\Omega)}{\chi}.
\end{align*}
Substituting the above bound and \eqref{eq:err.rho.1.a} into \eqref{eq:err.rho.1}, and using the definition of piecewise-broken $H^1$-seminorm and \eqref{eq:ID.approximation} yields the desired error estimate \eqref{eq:err.est.rho}.
\end{proof}


It remains to estimate the error for $\xi_h$. 
\begin{lemma}
    For any $\epsilon > 0$, there exists a positive constant dependent on $\epsilon$ and independent of $h$ such that
    \begin{align}
        \seminorm{H^1(\Omega)}{\xi - \xi_h} \lesssim h^{\min{((\pi / \omega) - \epsilon, k)}}. \label{eq:err.est.xi}
    \end{align}
\end{lemma}
\begin{proof}
        Given $\rho \in H^1(\Omega),$ let $\tilde{\xi}_{0,h} \in V_h^0$ solve the following discrete auxiliary problem, 
    \begin{align}\label{eq:P.xi0h.aux}
        a_h(\tilde{\xi}_{0,h}, \eta_h) + \beta(\Pi^0_{k,h} \tilde{\xi}_{0,h}, \Pi^0_{k,h} \eta_h) = (\rho, \Pi^0_{k,h} \eta_h) \quad \forall \eta_h \in V_h^0.
    \end{align}

    On comparing with \eqref{eq:Ph.xi0h} after using the definition of $\Pi^0_{k,h}$ on $(\Pi^0_{k,h} \rho_h, \Pi^0_{k,h} \eta_h)$, we have
    \begin{align*}
        a_h(\tilde{\xi}_{0,h} - \xi_{0,h}, \eta_h) + \beta (\Pi^0_{k,h}(\tilde{\xi}_{0,h} - \xi_{0,h}), \Pi^0_{k,h} \eta_h) = (\rho - \rho_h,\Pi^0_{k,h} \eta_h) \quad \forall \eta_h \in V_h^0.
    \end{align*}
    The right-hand side can be rewritten using the {\color{red} definition of $J_h$ in \eqref{def:Jh} as follows:
    \begin{align*}
        (\rho , J_h \eta_h) - (\rho_h, \Pi^0_{k,h} \eta_h) &= (\rho - \rho_h, J_h \eta_h) + (\rho_h, J_h \eta_h) - (\rho_h, \Pi^0_{k,h} \eta_h)  \\
        &= (\rho - \rho_h, J_h \eta_h).
    \end{align*}
    }

    It follows that 
    \begin{align}\label{eq:aux.est.xi0h}
        a_h(\tilde{\xi}_{0,h} - \xi_{0,h}, \eta_h) &\leq (\rho - \rho_h, J_h \eta_h) \quad \forall \eta_h \in V_h^0.
    \end{align}

    Testing \eqref{eq:aux.est.xi0h} with $\eta_h = \tilde{\xi}_{0,h} - \xi_{0,h} \in V_h^0$, using the dual norm estimate \eqref{eq:err.est.rho} with $\chi = J_h \eta_h$ and a Poincar\'e-Friedrichs inequality gives,
    \begin{align*}
        \norm{h}{\tilde{\xi}_{0,h} - \xi_{0,h}}^2 &\lesssim h^{\min{((\pi / \omega) - \epsilon, k)}} \seminorm{H^1(\Omega)}{J_h \eta_h} \norm{L^2(\Omega)}{\bs{f}}.
    \end{align*}
    Now the $H^1$ stability of the operator $J_h$ given in \eqref{eq:Jh.H1.stability} and Assumption \ref{assum:coercivity} lead to
    \begin{align}\label{eq:est.xi0h}
        \norm{h}{\tilde{\xi}_{0,h} - \xi_{0,h}} \lesssim h^{\min{((\pi / \omega) - \epsilon, k)}}\norm{L^2(\Omega)}{\bs{f}}
    \end{align}

    Since $\tilde{\xi}_{0,h}$ is the virtual element approximation of $\xi_0$ (see \eqref{eq:P.xi0} and \eqref{eq:P.xi0h.aux}), we can obtain the following error equation by testing with $J_h \eta_h$ in \eqref{eq:P.xi0} and subtracting it from \eqref{eq:P.xi0h.aux},
    \begin{align}\label{eq:erreq.xi0}
        a_h(\tilde{\xi}_{0,h}, \eta_h) - (\curl{\xi_0}, \curl{J_h \eta_h}) + \beta(\Pi^0_{k,h} \tilde{\xi}_{0,h}, \Pi^0_{k,h} \eta_h) - \beta(\xi_0, J_h \eta_h) = 0 \quad \forall \eta_h \in V_h^0.
    \end{align}
     The first difference in \eqref{eq:erreq.xi0} can be rewritten in controllable quantities as follows:
    \begin{align}\label{eq:err.xi0.1}
        &a_h(\tilde{\xi}_{0,h} - \tilde{\xi}_0, \eta_h) + a_h(\tilde{\xi}_0, \eta_h) - (\curl{\xi_0}, \curl{J_h \eta_h}) \nonumber\\
        &\quad= a_h(\tilde{\xi}_{0,h} - \tilde{\xi}_0, \eta_h) + \sum_{D \in \Th} (\curl{\Pi^1_{k,D} \xi_0}, \curl{\Pi^1_{k,D} \eta_h}) + S^D((I-\Pi^1_{k,D}) \xi_0, (I-\Pi^1_{k,D}) \eta_h) \nonumber\\
        &\qquad\qquad- (\curl{\xi_0}, \curl{J_h \eta_h}) \nonumber\\
        &\quad= a_h(\tilde{\xi}_{0,h} - \tilde{\xi}_0, \eta_h) + \sum_{D \in \Th} (\curl{(\Pi^1_{k,D} \xi_0} - \xi_0), \curl{\eta_h})+  S^D((I-\Pi^1_{k,D}) \xi_0, (I-\Pi^1_{k,D}) \eta_h) \nonumber\\
        &\qquad\qquad - (\curl{\xi_0}, \curl{(\eta_h - J_h \eta_h)}).
    \end{align}
    To estimate the last term in \eqref{eq:err.xi0.1} {\color{red} we require  that $\curl{(\eta_h - J_h \eta_h)} \perp \curl{\Poly_k{(\Th)}}$,
    \begin{align*}
        (\curl{\xi_0}, \curl{(\eta_h - J_h \eta_h)}) = (\curl{(\xi_0 - \Pi^0_{k,h} \xi_0)}, \curl{(\eta_h - J_h \eta_h)}).
    \end{align*}
    }
    The second difference in \eqref{eq:erreq.xi0} can be rewritten in controllable quantities as follows:
    \begin{align}\label{eq:err.xi0.2}
        &\beta(\Pi^0_{k,h} \tilde{\xi}_{0,h}, \Pi^0_{k,h} \eta_h) - \beta (\xi_0, \eta_h) - \beta(\xi_0, J_h \eta_h - \eta_h) 
    \end{align}
    The first difference in \eqref{eq:err.xi0.2} can be readily estimated using the definition of the $L^2$-projector to get,
    \begin{align*}
        \beta(\Pi^0_{k,h} \tilde{\xi}_{0,h}, \Pi^0_{k,h} \eta_h) - \beta (\xi_0, \eta_h) =\beta(\Pi^0_{k,h} (\tilde{\xi}_{0,h} - \xi_0), \Pi^0_{k,h} \eta_h) + \beta(\Pi^0_{k,h} \xi_0 - \xi_0,\eta_h).
    \end{align*}
    {\color{red} The last term in \eqref{eq:err.xi0.2} requires that $J_h \eta_h - \eta_h \perp \Poly_k(\Th)$,
    \begin{align*}   
        \beta(\xi_0, J_h \eta_h - \eta_h) = \beta(\xi_0 - \Pi^0_{k,h} \xi_0, J_h \eta_h - \eta_h).
    \end{align*}
    }
    \corr{}{}{
    [JT: The current operator $J_h$ is too strong to also satisfy the required orthogonality properties mentioned above. On going through the literature, we do have such a ``companion" operator defined in \cite[Theorem 2.5]{Khot-Nataraj-Verma-2023-Conforming-Companion-VEM} with the help of a simplicial sub-mesh. However that operator is too weak to also satisfy the orthogonality property of $J_h$. I am currently stuck here in the analysis.]  
    }
\end{proof}



