\documentclass[oneside, reqno, 11pt, a4paper]{amsart}

\usepackage[english]{babel}


%%%%%%%%%%
%% Fonts %%
%%%%%%%%%%
\AtBeginDocument{
  \DeclareSymbolFont{AMSb}{U}{msb}{m}{n}
  \DeclareSymbolFontAlphabet{\mathbb}{AMSb}
}
\DeclareFontFamily{U}{mathx}{\hyphenchar\font45}
\DeclareFontShape{U}{mathx}{m}{n}{<-> mathx10}{}
\DeclareSymbolFont{mathx}{U}{mathx}{m}{n}
\DeclareMathAccent{\widebar}{0}{mathx}{"73}

\usepackage{graphicx}
%\usepackage{subfig}
\usepackage{svg}
\usepackage{tikz}
\usepackage{caption}
\usepackage{aligned-overset}
\usepackage{subcaption}
\tikzstyle{arrow} = [thick,->,>=stealth]
\usetikzlibrary{shapes.misc,matrix,fit,positioning,arrows.meta,decorations.pathreplacing,calc,shapes.geometric,arrows,shapes,patterns,math}
\tikzset{Matrix/.style={matrix of nodes, font=\footnotesize,text height=1pt, text depth=0.5pt, text width=8.5pt, align=center, column sep=0pt, row sep=0pt, nodes in empty cells}}

\graphicspath{{Figures/}{Figures/results_3d/}{Figures/results/}}

%%%%%%%%%%%%
%% Acronyms %%
%%%%%%%%%%%%
\usepackage{acronym}
% Acronyms used in the paper
% \begin{acronym}
\acrodef{fem}[FEM]{Finite Element Method}
\acrodef{fems}[FEMs]{Finite Element Methods}
\acrodef{gmg}[GMG]{Geometric Multigrid Method}
\acrodef{gmgs}[GMGs]{Geometric Multigrid Methods}
\acrodef{ddm}[DDM]{Domain Decomposition Method}
\acrodef{ddms}[DDMs]{Domain Decomposition Methods}
\acrodef{bddc}[BDDC]{Balancing Domain Decomposition by Constraints}
\acrodef{feti}[FETI]{Finite Element Tearing and Interconnecting}
\acrodef{feti-dp}[FETI-DP]{Dual-Primal Finite Element Tearing and Interconnecting}
\acrodef{hho}[HHO]{Hybrid High-Order}
\acrodef{hdg}[HDG]{Hybridizable Discontinuous Galerkin}
\acrodef{vem}[VEM]{Virtual Element Method}
\acrodef{wg}[WG]{Weak Galerkin}
\acrodef{dg}[DG]{Discontinuous Galerkin}
\acrodef{bnn}[BNN]{Balancing Neumann-Neumann}
\acrodef{pde}[PDE]{Partial Differential Equation}
\acrodef{pdes}[PDEs]{Partial Differential Equations}
\acrodef{dof}[DOF]{Degree of Freedom}
\acrodef{dofs}[DOFs]{Degrees of Freedom}
% \end{acronym}


%%%%%%%%%%%%
%% Layout %%
%%%%%%%%%%%%
\usepackage[a4paper, pdftex, left=2cm, top=2cm, right=2cm, bottom=2cm]{geometry}
\usepackage[final]{pdfpages}
\usepackage{changepage}
\newcommand{\br}{\vskip 1em}
\usepackage[foot]{amsaddr}
\numberwithin{equation}{section}

%%%%%%%%%%%
%% Links %%
%%%%%%%%%%%
\usepackage{url}
\usepackage{hyperref}
\hypersetup{
    breaklinks=true,
    bookmarksopen=true,
    pdftitle={VEM for a Quad-Curl Problem}, 
    pdfauthor={S. C. Brenner, L. Y. Sung, J. Tushar}, 
	pdfauthor={}
    pdfsubject={}, 
    colorlinks=true,
    linkcolor=black,
    citecolor=blue,
    filecolor=black,
    urlcolor=blue
}
\newcommand{\fig}[1]{Fig.~\ref{#1}}
\newcommand{\tab}[1]{Tab.~\ref{#1}}
\newcommand{\sect}[1]{Sect.~\ref{#1}}
\newcommand{\app}[1]{Apdx.~\ref{#1}}

% Define correction command
 \usepackage{refcheck}
\usepackage[normalem]{ulem}
\normalem
\newcounter{corr}

%%%%%%%%%%%%%%
%% Graphics %%
%%%%%%%%%%%%%%
% \usepackage[dvipsnames]{xcolor}
\definecolor{RedOrange}{rgb}{1,0.3,0}
\definecolor{violet}{rgb}{0.580,0.,0.827}  
\newcommand{\corr}[3]{\typeout{Warning : a correction remains in page \thepage}
  \stepcounter{corr}        
	      {\color{blue}\ifmmode\text{\,\sout{\ensuremath{#1}}\,}\else\sout{#1}\fi}
              {\color{red}#2}
              {\color{violet} #3}
}

% \newcommand{\corr}[3]{{\color{RedOrange}\bf $\blacktriangleright$ #3 $\blacktriangleleft$}}

% Revision highlighting (toggle with \revmodefalse to hide)
\newif\ifrevmode
\revmodetrue
\newcommand{\revadd}[1]{\ifrevmode{\color{red}#1}\else#1\fi}
\newcommand{\revdel}[1]{\ifrevmode{\color{blue}\ifmmode\text{\,\sout{\ensuremath{#1}}\,}\else\sout{#1}\fi}\else\ignorespaces\fi}
\newcommand{\revrep}[2]{\ifrevmode{\revdel{#1}{}}{\color{red}#2}\else#2\fi}

%%%%%%%%%%%%%%%%%%
%% Bibliography %%
%%%%%%%%%%%%%%%%%%
\usepackage{csquotes}
\usepackage[backend=biber, defernumbers=true, maxbibnames=5, style=numeric-comp, isbn=false, bibencoding=utf8, safeinputenc, url=false, doi=true, giveninits=true]{biblatex}
\addbibresource{refs.bib}

%%%%%%%%%%%%%%%%%%
%% Environments %%
%%%%%%%%%%%%%%%%%%
\usepackage{mathtools}
\usepackage{mathabx}

\usepackage{float}
\usepackage{framed}
\usepackage{verbatim}
\usepackage{fancyvrb}
\usepackage{booktabs}
\usepackage{colortbl}
\usepackage{multirow}
\usepackage{algorithm}
\usepackage{algpseudocode}
\usepackage{cancel}
\usepackage{accents}
\usepackage{mleftright}
\usepackage{thm-restate}
\usepackage{longtable}

\makeatletter
\algdef{S}[IF]{IfNoThen}[1]{\algorithmicif\ #1}
\makeatother
\usepackage[inline]{enumitem}
\usepackage{siunitx}


\newtheorem{theorem}{Theorem}
\numberwithin{theorem}{section}
\newtheorem{remark}[theorem]{Remark}
\newtheorem{lemma}[theorem]{Lemma}
\newtheorem{corollary}[theorem]{Corollary}
\newtheorem{proposition}[theorem]{Proposition}
\newtheorem{assumption}[theorem]{Assumption}
\usepackage[breakable]{tcolorbox}
\tcbuselibrary{theorems}
\tcbuselibrary{skins}
\tcbset{
	commonstyle/.style={
		theorem style=plain,
		enhanced jigsaw,
		fonttitle=\bfseries,
		fontupper=\itshape,
		halign=justify,
		separator sign=:,
		description delimiters none,
		description font=\bfseries, 
		terminator sign={.\hspace{0.25em}},
		arc=0mm,outer arc=0mm,
		boxrule=0pt,toprule=0pt,bottomrule=0pt,leftrule=0pt,rightrule=0pt,
		titlerule=0pt,toptitle=0pt,bottomtitle=0pt,top=0pt,
		colback=white,coltitle=black,
		boxsep=0pt, bottom=0pt, left=0pt, 
	}
}
\newtcbtheorem[]{myproblem}{Problem}%
{center, commonstyle, fonttitle=\bfseries}{pb}
\newtcbtheorem[]{mydefinition}{Definition}
{center, commonstyle, fonttitle=\bfseries}{pb}
\newtcbtheorem[]{myassumption}{Assumption}
{center, commonstyle, fonttitle=\bfseries}{pb}

%%%%%%%%%%%%%%%%%
%% Mathematics %%
%%%%%%%%%%%%%%%%%
\usepackage{amsmath, amsfonts, amssymb, amscd, bm, mathtools}

% Mesh (micro, macro)

\newcommand{\R}{\mathbb{R}}
\newcommand{\E}{\mathcal{E}}
\newcommand{\curl}[1]{\nabla \times #1}
\newcommand{\grad}[1]{\nabla #1}
\newcommand{\bs}[1]{\boldsymbol{#1}}
\newcommand{\divr}[1]{\mbox{div}\,#1}
\newcommand{\Th}{\mathcal{T}_h}
\newcommand{\ED}{\mathcal{E}_D}
\newcommand{\Poly}{\mathbb{P}}
\newcommand{\norm}[2]{\|#2\|_{#1}}
\newcommand{\A}{\mathcal{A}}
\newcommand{\seminorm}[2]{|#2|_{#1}}
\newcommand{\Ih}[1]{I_{k,h} #1}
\newcommand{\ID}[1]{I_{k,D} #1}
\newcommand{\tnorm}[2]{\vvvert #2\vvvert_{#1}}

% \newcommand{\mh}{\mathbb{M}_{h}}
% \newcommand{\tah}{\mathbb{T}_{h}}
% \newcommand{\fh}{\mathbb{F}_{h}}
% \newcommand{\Mh}{\mathcal{M}_{h}}
% \newcommand{\Th}{\mathcal{T}_{h}}
% \newcommand{\Thloc}{\mathcal{T}^i_{h}}
% \newcommand{\Fh}{\mathcal{F}_{h}}
% \newcommand{\Fhloc}{\mathcal{F}^i_{h}}
% \newcommand{\interior}{{\rm in}}
% \newcommand{\boundary}{{\rm bd}}
% \newcommand{\Fhi}{\Fh^{\interior}}
% \newcommand{\Fhb}{\Fh^{\boundary}}
% \newcommand{\Fhbloc}{\Fh^{\boundary,i}}
% \newcommand{\FFhb}{\mathcal{F\!F}_h^{\boundary}}
% \newcommand{\FFhbloc}{\mathcal{F\!F}_h^{\boundary,i}}
% \newcommand{\FFh}{\mathcal{F\!F}_h}
% \newcommand{\Eh}{\mathcal{E}_{h}}
% \newcommand{\Ehi}{\Eh^{\interior}}
% \newcommand{\Ehb}{\Eh^{\boundary}}
% \newcommand{\MH}{\mathcal{M}_{H}}
% \renewcommand{\TH}{\mathcal{T}_{H}}
% \newcommand{\FH}{\mathcal{F}_{H}}
% \newcommand{\FHi}{\FH^{\interior}}
% \newcommand{\FHb}{\FH^{\boundary}}
% % Projectors
% \newcommand{\lproj}[2]{\pi_{#1}^{#2}}
% %\newcommand{\lproj}[2]{\pi_{#1}^{0,#2}}
% %\newcommand{\eproj}[2]{\pi_{#1}^{1,#2}}
% \newcommand{\projFace}{\mathfrak{F}}
% % To distinguish continuous and discontinuous vectors/spaces
% \newcommand{\wh}[1]{\widehat{#1}}
% \newcommand{\wt}[1]{\widetilde{#1}} 
% \newcommand{\ul}[1]{\underline{#1}}
% % HHO space: continuous (interior and boundary), discontinuous, micro, macro
% \newcommand{\Uhbddc}{\ul{\wt{U}}_h}
% \newcommand{\Uhbub}{\ul{U}_{h,0}}
% \newcommand{\Uhbubb}{U_{h,0}^\partial}
% \newcommand{\Uhc}{\wh{\ul{U}}_{h}}
% % \newcommand{\Uh}{\ul{U}_{h}}
% \newcommand{\Uh}{\Uhc}
% \newcommand{\Uhci}{\wh{U}_{h}}
% \newcommand{\Uhi}{U_{h}}
% \newcommand{\Uhcloc}{\wh{\ul{U}}^i_{h}}
% \newcommand{\Uhloc}{\ul{U}^i_{h}}
% \newcommand{\Uhcb}{\wh{U}_{h}^{\partial}}
% % \newcommand{\Uhb}{U_{h}^{\partial}}
% \newcommand{\Uhb}{\Uhcb}
% \newcommand{\Uhcbb}{\wh{U}_{h}^{\partial,\boundary}}
% % \newcommand{\Uhbb}{U_{h}^{\partial,\boundary}}
% \newcommand{\Uhbb}{\Uhcbb}
% \newcommand{\bb}{{\partial,\boundary}}
% \newcommand{\Uhcbloc}{\wh{U}_{h}^{\partial,i}}
% \newcommand{\Uhbloc}{U_{h}^{\partial,i}}
% \newcommand{\Uhcbbloc}{\wh{U}_{h}^{\partial,\boundary,i}}
% \newcommand{\Uhbbloc}{U_{h}^{\partial,\boundary,i}}
% \newcommand{\bs}[1]{\boldsymbol{#1}}

% \newcommand{\Uhd}{\ul{U}_{h}}
% \newcommand{\Uhdi}{U_{h}}
% \newcommand{\Uhdb}{U_{h}^{\partial}}

% \newcommand{\UHc}{\wh{\ul{U}}_{H}}
% \newcommand{\UHci}{\wh{U}_{H}}
% \newcommand{\UHcb}{\wh{U}_{H}^{\partial}}
% \newcommand{\UHd}{\ul{U}_{H}}
% \newcommand{\UHdi}{U_{H}}
% \newcommand{\UHdb}{U_{H}^{\partial}}
% \newcommand{\Uhei}{\wh{\mathcal{U}}_{h}}

% \newcommand{\Iht}{\ul{I}_{t,h}}
% \newcommand{\Ih}{\ul{I}_h}

% \newcommand{\ctr}{\gamma}
% \newcommand{\tr}{\gamma}

% \newcommand{\fbou}{f}

% \newcommand{\fF}{{f\!F}}
% \newcommand{\ffp}{{f \! f' }}
% \newcommand{\ff}{{f\!f}}
% % \newcommand{\dffp}{\delta_{\ffp}}
% \newcommand{\dffp}{|x_f-x_{f'}|}
% \newcommand{\dfg}{|x_f-x_{g}|}
% % \newcommand{\dffp}{\delta(f,f')}
% \newcommand{\dff}{\delta_\ff}
% \newcommand{\fGbd}{{f\partial\Gamma}}
% \newcommand{\dfGbd}{\mbox{dist}(x_f,\partial\Gamma)}
% \newcommand{\dist}{\mathop{\rm dist}}
% \newcommand{\distV}{\mathop{\rm dist}\nolimits_V}
% \newcommand{\distH}{\mathop{\rm dist}\nolimits_H}
% %\newcommand{\vvvert}{\vert\kern-0.25ex\vert\kern-0.25ex\vert}
% % \newcommand{\tnorm}[2]{\vvvert #2\vvvert_{#1}}
% 
% \newcommand{\Poly}[1]{\mathbb{P}^{#1}}
 
% % Domain
% % \newcommand{\dom}{\mathcal{O}}
% \newcommand{\dom}{\Omega}

% % Annulus
% \newcommand{\An}[3]{{\ocirc}_{#2}^{#1}(#3)}
% %\newcommand{\An}[3]{{\textstyle\bigodot}_{#1,#2}(#3)}
% % \newcommand{\R}{\mathcal{R}_{fl}}
% \newcommand{\Above}{\mathcal{V}}

% % Disc
% \newcommand{\disc}[2]{\ovoid_{#1}(#2)}

% % average
% \newcommand{\ol}[1]{\overline{#1}}

% % Slicings
% \newcommand{\A}{\mathcal{A}}

% %Neighbour
% \newcommand{\N}{\mathcal{N}}

% %vertical Layers
% \newcommand{\La}{\mathcal{L}}
% \newcommand{\G}{\mathcal{G}}

% %Horizontal layers
% \newcommand{\Co}{\mathcal{C}}
% \newcommand{\Ca}{\mathcal{C}_{\ffp s}}

% %
% \newcommand{\W}{\mathcal{W}}

% %I_{ff'}
% \newcommand{\Iffp}{\mathcal{I}_{\ffp}}
% \newcommand{\Ift}{\mathcal{I}_{ft}}
% % \newcommand{\Iffp}{\mathcal{I}({\fbou , \fbou') }}
% % \newcommand{\Ift}{\mathcal{I}{(\fbou,t)}}
% \newcommand{\fint}{g}

% %Operators
% \newcommand{\HoneOp}{\boldsymbol{\mathcal{A}}}
% \newcommand{\HhalfOp}{\boldsymbol{\mathcal{H}}}
% \newcommand{\U}{\mathcal{U}}
% \newcommand{\V}{\mathcal{V}}
% \newcommand{\Q}{\boldsymbol{\mathcal{Q}}}


% %Disjoint Union
% \newcommand{\rotpi}{\rotatebox[origin=c]{180}{\Large $\Pi$}}


%%%%%%%%%%%%%%
%% Metadata %%
%%%%%%%%%%%%%%
\title[]{
Virtual Element Method for a Quad-Curl Problem on Planar Domains
} 
\date{\today}
\keywords{}

\address{$^\dagger$ Department of Mathematics and Center for Computation \& Technology\\Louisiana State University\\Baton Rouge\\Louisiana 70803\\USA}
\author[S. C. Brenner]{Susanne C. Brenner$^{\dagger}$}
% \thanks{\textsuperscript{*}Corresponding author.}
\email{brenner@math.lsu.edu}
\author[L.-Y. Sung]{Li-Yeng Sung$^{\dagger}$}
\email{sung@math.lsu.edu}
\author[J. Tushar]{Jai Tushar$^{\dagger}$}
\email{Jai.Tushar@lsu.edu}

\begin{document}


	\begin{abstract}
		$\dots$  
	\end{abstract}

	

	\keywords{$\dots$}

	\maketitle

	% Introduction
	\section{Introduction}\label{sec:introduction}

Consider a bounded polygonal domain $\Omega \subset \R^2$ with boundary $\partial \Omega$. We are interested in the following quad-curl problem,
\begin{align}\label{eq:P.strong}
    \begin{aligned}
    (\curl{})^4\bs{u} + \beta \; (\curl{})^2\bs{u} + \gamma \; \bs{u} &= \bs{f} \quad \mbox{in} \quad \Omega, \\
    \curl{\bs{u}} &= 0 \quad \mbox{on} \quad \partial \Omega, \\
    \bs{n} \times \bs{u} &= 0 \quad \mbox{on} \quad \partial \Omega.
    \end{aligned}
\end{align} 
Here $(\curl{\cdot})^4 = \curl(\curl(\curl(\curl \cdot)))$ and $(\curl{\cdot})^2 = \curl(\curl \cdot)$, are the quad-curl and curl-curl operators in two dimensions, $\bs{n}$ is the unit outward normal vector on $\partial \Omega$, and $\beta, \gamma \geq 0$ are given constants with $\gamma > 0$, if $\Omega$ is multiply connected. The forcing term $\bs{f}: \Omega \to \R^2$ is a given function.

The weak formulation of \eqref{eq:P.strong} reads: find $\bs{u} \in \tilde{\E}$ such that
\begin{align}\label{eq:P}
    (\curl(\curl \bs{u}), \curl(\curl \bs{v})) + \beta (\curl \bs{u}, \curl \bs{v}) + \gamma (\bs{u}, \bs{v}) = (\bs{f}, \bs{v}) \quad \forall \bs{v} \in \tilde{\E},
\end{align} 
where the energy space $\tilde{\E}$ is defined as
\begin{align*}
    \tilde{\E} := \left\{ \bs{v} \in [L^2(\Omega)]^2: \curl{\bs{v}} \in H_0^1(\Omega), \;\; \bs{n} \times \bs{v} = 0 \;\; \mbox{on} \;\; \partial\Omega \right\}.
\end{align*}
 The notation $(\cdot, \cdot)$ denotes the $L^2(\Omega)$ (or $[L^2(\Omega)]^2$) inner-product. This quad-curl problem is linked to the Maxwell's transmission eigenvalue problem (see (2.8) and consider for e.g., the case of homogeneous media in \cite{Monk-Sun-2012-Maxwell-Curl-Conforming}). The authors here opt for a curl-conforming finite element method to discretize the problem. The quad-curl problem also has several applications including the resistive magnetohydrodynamic (MHD) \dots \corr{}{}{[JT: To do - Add more literature]}.

\smallskip

Another popular approach to numerically solve \eqref{eq:P.strong} is to use the Hodge decomposition of divergence-free vector fields (see \eqref{eq:HodgeDecomposition} in Section \ref{sec:Reduction}). This reduces the problem into a sequence of second-order problems, which can then be discretized using $H^1$-conforming finite element methods. This approach has been studied, for e.g., in \cite{Brenner-Sun-Sung-2017-HodgeDecomposition2D, Brenner-Cavanaugh-Sung-2024-HodgeDecomposition3D}  in two and three dimensions using simplicial and tetrahedral meshes, respectively. Since, we seek the solution in the divergence-free space (see Remark \ref{rem:kernel}), we solve \eqref{eq:P} using the reduced energy space $\E \subset \tilde \E$,
\begin{align}\label{eq:E}
    \E := \left\{ \bs{v} \in [L^2(\Omega)]^2: \curl{\bs{v}} \in H_0^1(\Omega), \;\; \divr{\bs{v}} = 0 \;\; \mbox{and} \;\; \bs{n} \times \bs{v} = 0 \;\; \mbox{on} \;\; \partial\Omega \right\}.
\end{align}

\smallskip

Our goal in this paper is to extend the results to polygonal meshes using the conforming virtual element method (VEM). \corr{}{}{[JT: To do - Add literature on VEM for Maxwell and Hodge decomposition]}


	% Preliminaries
	\section{Preliminaries}\label{sec:preliminaries}

In two dimensions, for a vector field $\bs{v} = (v_1, v_2)$, the curl operator is a scalar function,
\begin{align*}
    \curl{\bs{v}} = \frac{\partial v_2}{\partial x} - \frac{\partial v_1}{\partial y}.
\end{align*}
For a scalar function $\phi$, the curl operator is a vector field,
\begin{align*}
    \curl{\phi} = \left[ \frac{\partial \phi}{\partial y}, -\frac{\partial \phi}{\partial x} \right]^\top = \mbox{rot} \; \phi.
\end{align*}
As a result, we have 
\begin{align*}
    (\curl{\phi}, \curl{v}) = (\grad{\phi},\grad{v}) \quad \forall \phi, v \in H^1(\Omega),
\end{align*}
and $\norm{L^2(\Omega)}{\curl{v}} = \seminorm{H^1(\Omega)}{v}$ for all $v \in H^1(\Omega)$.

\smallskip

We collect regularity results for elliptic Boundary Value Problems (BVPs) on polygonal domains (see \cite[Section 5.1]{grisvard:1985:elliptic} and \cite[Section 2.5]{Dauge-1988-Regularity-Corner-Domains}), with $\omega$ denoting the largest interior angle at the corners of $\Omega$.

\begin{lemma}[Regularity of reaction-diffusion BVP]\label{lem:reg.reac-diff}
    Given $g \in H^1(\Omega)$, let $\mu \in H_0^1(\Omega)$ satisfy
    \begin{align*}
        (\grad{\mu}, \grad{v}) + \beta (\mu, v) = (g, v) \quad \forall v \in H_0^1(\Omega).
    \end{align*}
    Then for any $\epsilon > 0, \mu \in H^{1+(\pi / \omega) - \epsilon}(\Omega)$. Also, we have the following continuous dependence on the data result, with positive constant $C_\epsilon$, depending on $\epsilon$ and $\Omega$,
    \begin{align*}
        \norm{H^{1+(\pi / \omega) - \epsilon}(\Omega)}{\mu} \leq C_\epsilon \norm{H^1(\Omega)}{g}.
    \end{align*}
\end{lemma}

\begin{lemma}[Regularity of pure-Neumann BVP]\label{lem:reg.pure-neumann}
    Given $g \in H^1(\Omega)$, let $\lambda \in H^1(\Omega)$ satisfy
    \begin{align*}
        (\grad{\lambda}, \grad{v}) + (\lambda, 1) (v,1) = (g, v) \quad \forall v \in H^1(\Omega).
    \end{align*}
    Then for any $\epsilon > 0, \lambda \in H^{1+(\pi / \omega) - \epsilon}(\Omega)$. Also, we have the following continuous dependence on the data result, with positive constant $C_\epsilon$, depending on $\epsilon$ and $\Omega$,
    \begin{align*}
        \norm{H^{1+(\pi / \omega) - \epsilon}(\Omega)}{\lambda} \leq C_\epsilon \norm{H^1(\Omega)}{g}.
    \end{align*}
\end{lemma}

Using the above lemmas, we now state the regularity result for the energy space $\E$ defined in \eqref{eq:E} (see \cite[Section 2.1]{Brenner-Sun-Sung-2017-HodgeDecomposition2D} for a proof).
\begin{theorem}[Regularity of the energy space]\label{thm:reg.E} 
    The quad-curl energy space $\mathbb{E}$ defined in \eqref{eq:E} has 
    regularity $[H^{(\pi / \omega)-\epsilon}(\Omega)]^2$ for any 
    $\epsilon > 0$.
\end{theorem}
As a consequence of this, the authors in \cite[Section 2.2]{Brenner-Sun-Sung-2017-HodgeDecomposition2D} established the well-posedness of the quad-curl problem \eqref{eq:P} in the energy space $\E$.

\smallskip

We will use following Hilbert spaces in the rest of the article,
\begin{align*}
    H(\divr{}^0; \Omega) &:= \left\{ \bs{v} \in [L^2(\Omega)]^2: (\bs{v}, \grad \eta) = 0 \quad \forall \eta \in H_0^1(\Omega) \right\}, \\
    % H_0(\mbox{curl};\Omega) &:= \left\{ \bs{v} \in [L^2(\Omega)]^2: \curl{\bs{v}} \in L^2(\Omega), \;\; \bs{n} \times \bs{v} = 0 \;\; \mbox{on} \;\; \partial\Omega\right\}, \\ 
    L^2_0(\Omega) &:= \left\{ v \in L^2(\Omega): (1,v) = 0 \right\}.
\end{align*}



% $$
% ,
% $$
% and the space of zero mean value functions by 
% $$
% L^2_0(\Omega) := \left\{ v \in L^2(\Omega): (1,v) = 0 \right\}.
% $$

	% % Hodge Decomposition and Energy Space
	% \section{Hodge decomposition and the energy space $\mathbb{E}$}\label{sec:Hd.E}

Any $\bs{v} \in H(\divr{}^0;\Omega)$ has a unique decomposition (see Chapter 3 in \cite{Girault-Raviart-1986-FEM-NS} and (1.2) in \cite{Brenner-Cui-Nan-Sung-2012-HodgeDecomposition2D}):
\begin{align}\label{eq:HodgeDecomposition}
    \bs{v} = \curl{\phi} + \sum_{j = 1}^{m} d_j \grad \varphi_j,
\end{align}
where $\phi \in H^1(\Omega) \cap L^2_0(\Omega)$, $m \in \mathbb{Z}^+_0$ is the Betti number, and $d_j (1 \leq j \leq m) \in \R$. The harmonic functions $\varphi_j$ are defined by 
\begin{subequations}\label{eq:P.varphi}
    \begin{align}
        (\grad \varphi_j, \grad v) &= 0 \quad \forall v \in H_0^1(\Omega), \label{eq:P.varphi.a}\\
        \varphi_j|_{\partial \Omega} &= 0, \label{eq:P.varphi.b}\\
        \varphi_j|_{\Gamma_l} &= \delta_{jl} \quad \mbox{for} \quad 1 \leq l \leq m.\label{eq:P.varphi.c}
    \end{align}
\end{subequations}
Here $\Gamma_l$ denote the $l$ components of the inner boundary when Betti number $m > 0$, and $\delta_{jl}$ is the Kronecker delta function.

\medskip

For any $\bs{v} \in \mathbb{E} \subset H(\divr{}^0;\Omega)$, we can apply the Hodge decomposition \eqref{eq:HodgeDecomposition} and the orthogonality it offers, for all $\rho \in H_0^1(\Omega)$, to get
\begin{align*}
    (\curl{\bs{v}}, \curl \rho) = (\curl{\phi}, \curl \rho).
\end{align*}
Since $\mathbb{E} \subset H_0(curl;\Omega)$, we also have via an integration by parts for $\curl{}$ identity that
\begin{align*}
    (\curl{\bs{v}}, \curl \rho) = (\bs{v}, \curl \rho).
\end{align*}
This implies that for any $\phi \in H^1(\Omega) \cap L_0^2(\Omega)$,
% \begin{align}\label{eq:}
    
% \end{align}



	% Reduction
	\section{Hodge Decomposition Based Reduction}\label{sec:Reduction}

Any $\bs{v} \in H(\divr{}^0;\Omega)$ has a unique decomposition (see Chapter 3 in \cite{Girault-Raviart-1986-FEM-NS} and (1.2) in \cite{Brenner-Cui-Nan-Sung-2012-HodgeDecomposition2D}):
\begin{align}\label{eq:HodgeDecomposition}
    \bs{v} = \curl{\phi} + \sum_{j = 1}^{m} d_j \grad \varphi_j,
\end{align}
where $\phi \in H^1(\Omega) \cap L^2_0(\Omega)$, $m \in \mathbb{Z}^+_0$ is the Betti number, and $d_j (1 \leq j \leq m) \in \R$. The harmonic functions $\varphi_j$ are defined by 
\begin{subequations}\label{eq:P.varphi}
    \begin{align}
        (\grad \varphi_j, \grad v) &= 0 \quad \forall v \in H_0^1(\Omega), \label{eq:P.varphi.a}\\
        \varphi_j|_{\Gamma_0} &= 0, \label{eq:P.varphi.b}\\
        \varphi_j|_{\Gamma_l} &= \delta_{jl} \quad \mbox{for} \quad 1 \leq l \leq m.\label{eq:P.varphi.c}
    \end{align}
\end{subequations}
Here $\Gamma_0$ denotes the outer boundary, $\Gamma_l$ denote the $l$ components of the inner boundary when Betti number $m > 0$, and $\delta_{jl}$ is the Kronecker delta function.


\begin{remark}\label{rem:kernel}
  The quad-curl energy has a large kernel consisting of gradient fields. Since the functions in $H( \divr{}^0; \Omega)$ are orthogonal to gradient fields, this kernel is fixed upto harmonic functions. Furthermore, in the case of simply connected domains, we infer from the boundary condition $\bs{u} \times \bs{n} = 0$ that the kernel reduces to the zero vector field. Therefore, we can put $\gamma = 0$. However, in the case of multiply connected domains, the harmonic functions in the kernel are non-zero vector fields. To account for this, we set $\gamma > 0$.
\end{remark}

\subsection{$\Omega$ is simply connected ($\gamma$ = 0).}
The Hodge decomposition \eqref{eq:HodgeDecomposition} reduces to 
$$ \bs{u} = \curl{\phi}. $$
 It now remains to find $\phi$. As shown in \cite{Brenner-Sun-Sung-2017-HodgeDecomposition2D}, this can be achieved by solving the following sequence of problems. 

\begin{enumerate}
    \item First we find $\rho \in H^1(\Omega) \cap L_2^0(\Omega)$, or equivalently $\rho \in H^1(\Omega)$ such that
          \begin{align}\label{eq:P.rho}
            (\curl{\rho}, \curl{\psi}) + (\rho,1) (\psi,1) = (\bs{f}, \curl{\psi}) \quad \forall \psi \in H^1(\Omega).
          \end{align}

    \item Find $\xi \in H^1(\Omega) \cap L_2^0(\Omega)$ given by 
          \begin{align}\label{eq:P.xi}
            \xi = \xi_0 - \frac{(1,\xi_0)}{(1, \xi_1)} \xi_1, 
          \end{align}
          where $\xi_0, \xi_1 \in H_0^1(\Omega)$ satisfy 
          \begin{align}
            (\curl{\xi_0}, \curl{\eta}) + \beta(\xi_0, \eta) = (\rho, \eta) \quad \forall \eta \in H_0^1(\Omega), \label{eq:P.xi0}\\
            (\curl{\xi_1}, \curl{\eta}) + \beta(\xi_1, \eta) = (1, \eta) \quad \forall \eta \in H_0^1(\Omega). \label{eq:P.xi1}
          \end{align}

    \item Finally we find $\phi \in H^1(\Omega) \cap L_2^0(\Omega)$, or equivalently $\phi \in H^1(\Omega)$ such that
          \begin{align}\label{eq:P.phi}
            (\curl{\phi}, \curl{\psi}) + (\phi,1) (\psi,1) = (\xi, \psi) \quad \forall \psi \in H^1(\Omega).
          \end{align}
\end{enumerate}

\subsection{$\Omega$ is multiply connected ($\gamma > 0$).}
Recalling the Hodge decomposition \eqref{eq:HodgeDecomposition} 
$$ \bs{u} = \curl{\phi} + \sum_{j = 1}^{m} c_j \grad \varphi_j. $$
It remains to find $\phi, c_j,$ and $\varphi_j$. Following \cite{Brenner-Sun-Sung-2017-HodgeDecomposition2D}, this is equivalent to solving the following sequence of problems. 

\begin{enumerate}
    \item Find $(\zeta, \xi) \in H^1(\Omega) \times H_0^1(\Omega)$ given by
     \begin{align}\label{eq:P.zeta.chi}
        (\zeta, \xi) = (\zeta_0, \xi_0) - \frac{(1,\xi_0)}{(1, \xi_1)} (\zeta_1, \xi_1),
     \end{align}
     where $(\zeta_0, \xi_0), (\zeta_1, \xi_1) \in H^1(\Omega) \times H_0^1(\Omega)$ solve the following two coupled system:
     \begin{align}
            \A((\zeta_0,\xi_0),(\psi,\eta)) + (\zeta_0, 1) (\psi,1) &= \gamma^{-\frac{1}{2}}(\bs{f}, \curl{\psi}) \quad \forall (\psi, \eta) \in H^1(\Omega) \times H_0^1(\Omega), \label{eq:P.zeta0.xi0}\\
            \A((\zeta_1,\xi_1),(\psi,\eta)) + (\zeta_1, 1) (\psi,1) &= (1,\eta) \quad \forall (\psi, \eta) \in H^1(\Omega) \times H_0^1(\Omega). \label{eq:P.zeta1.xi1}
      \end{align}
      Here, the bilinear form $\A(\cdot, \cdot)$ is defined by
      \begin{align}\label{eq:A}
        \A((\zeta,\xi),(\psi,\eta)) = (\curl \zeta, \curl \psi) + \gamma^{\frac{1}{2}} (\psi, \xi) - \gamma^{\frac{1}{2}} (\zeta, \eta) + (\curl \xi, \curl \eta) + \beta(\xi, \eta).
      \end{align}


    \item Find $\phi \in H^1(\Omega)$ such that \eqref{eq:P.phi} holds.

    \item Finally $c_j (1 \leq j \leq m)$, are determined by solving the $m \times m$, SPD system
          \begin{align}\label{eq:P.cj}
            \sum_{j=1}^m (\grad \varphi_i, \grad \varphi_j) c_j = \gamma^{-1} (\bs{f}, \grad \varphi_i) \quad \mbox{for} \quad 1 \leq i \leq m,
          \end{align}
          and the harmonic functions $\varphi_j$ are defined by \eqref{eq:P.varphi}.
\end{enumerate}

	% Virtual Element Scheme
	\section{Conforming Virtual Element Discretization}\label{sec:discretization}

Let $\Th$ be a triangulation of the polygonal domain $\Omega \subset \R^2$ into a finite collection of simple polygons $D$. The mesh parameter $h$ is defined as $h := \max_{D \in \Th} h_D$, where $h_D$ denotes the diameter of $D$. Let $\ED$ be the set of edges associated with $D$. We make the following shape-regularity assumptions for all $D \in \Th$ (see \cite{Brenner-Guan-Sung-2017-Some-Estimates-VEM, beirao-da-veiga.brezzi.ea:2013:basic}). There exists a constant $\Theta \in (0,1)$ such that
\begin{subequations}\label{meshreg_assum}
    \begin{align}
        &\mbox{$D$ is star-shaped with respect to a ball of radius $\Theta h_D$, and}  \label{meshreg_assum.1}\\
        &\mbox{$|e| \geq \Theta h_D$ for any edge $e \in \ED$.} \label{meshreg_assum.2}
    \end{align}
\end{subequations}
The local enhanced virtual element space $V_h(D) \subset H^1(D)$ (see \cite{ahmad.alsaedi.ea:2013:equivalent,Brenner-Guan-Sung-2017-Some-Estimates-VEM}) is defined as follows:
\begin{align}\label{eq:VhK}
    V_h(D) := \left\{ v_h \in H^1(D): v_h|_{\partial D} \in \Poly_{k}(\partial D), \;\; -\Delta v_h \in \Poly_{k}(D), \;\; \Pi^0_{k,D} v_h - \Pi^{1}_{k,D} v_h \in \Poly_{k-2}(D)  \right\}.
\end{align}
Here $\Poly_k(\partial D)$ (respectively, $\Poly_k(D)$) denote the space of continuous piecewise polynomials of degree at most $k$ on the boundary $\partial D$ (respectively, on $D$). The operators $\Pi^0_{k,D}$ and $\Pi^{1}_{k,D}$ are the standard $L^2$ and $H^1$-projections onto $\Poly_k(D)$, respectively (see \cite[Section 2.2]{Brenner-Guan-Sung-2017-Some-Estimates-VEM} for details). The global virtual element spaces $V_h$ and $V_h^0$ are defined by concatenating the local spaces as follows:
\begin{align}
    V_h := \{ v_h \in H^1(\Omega) : v_h|_D \in V_h(D) \text{ for all } D \in \Th \}, \label{eq:Vh} \\
    V_h^0 := \{ v_h \in H_0^1(\Omega) : v_h|_D \in V_h(D) \text{ for all } D \in \Th \}. \label{eq:Vh0}
\end{align}

\subsection{$\Omega$ is simply connected ($\gamma = 0$).}
The $\Poly_k$ Virtual Element Method for approximating the sequence of problems \eqref{eq:P.rho}-\eqref{eq:P.phi} is as follows.

\begin{enumerate}
    \item Find $\rho_h \in V_h$ such that
          \begin{align}\label{eq:Ph.rhoh}
            a_h(\rho_h, \psi_h) + (\Pi^0_{k,h}\rho_h,1) (\Pi^0_{k,h} \psi_h,1) = (\bs{f}, \curl{ \Pi^{1}_{k,h} \psi_h}) \quad \forall \psi_h \in V_h.
          \end{align}

    \item Find $\xi_h \in V_h$ given by 
          \begin{align}\label{eq:Ph.xih}
            \xi_h = \xi_{0,h} - \frac{(1,\xi_{0,h})}{(1, \xi_{1,h})} \xi_{1,h}, 
          \end{align}
          where $\xi_{0,h}, \xi_{1,h} \in V_h^0$ satisfy 
          \begin{align}
            a_h(\xi_{0,h}, \eta_h) + \beta(\Pi^0_{k,h} \xi_{0,h}, \Pi^0_{k,h} \eta_h) &= (\Pi^0_{k,h}\rho_h, \Pi^0_{k,h} \eta_h) \quad \forall \eta_h \in V_h^0, \label{eq:Ph.xi0h}\\
            a_h(\xi_{1,h}, \eta_h) + \beta(\Pi^0_{k,h}\xi_{1,h},\Pi^0_{k,h} \eta_h) &= (1, \Pi^0_{k,h} \eta_h) \quad \forall \eta_h \in V_h^0. \label{eq:Ph.xi1h}
          \end{align}

    \item Find $\phi_h \in V_h$ such that
          \begin{align}\label{eq:Ph.phih}
            a_h(\phi_h, \psi_h) + (\Pi^0_{k,h}\phi_h,1) (\Pi^0_{k,h} \psi_h,1) = (\Pi^0_{k,h}\xi_h, \Pi^0_{k,h} \psi_h) \quad \forall \psi_h \in V_h.
          \end{align}
\end{enumerate}

Here, the global bilinear form $a_h(\cdot, \cdot)$ is given elementwise by:
\begin{align}
    a_h(w_h, v_h) &= \sum_{D \in \Th} a^D_h(w_h, v_h) \quad \forall w_h, v_h \in V_h, \nonumber\\
                   &= \sum_{D \in \Th}  (\curl{\Pi^{1}_{k,D} w_h}, \curl{\Pi^{1}_{k,D} v_h})_D + S^D((I - \Pi^{1}_{k,D}) w_h, (I - \Pi^{1}_{k,D}) v_h). \label{eq:ah}
\end{align}

$\Pi^0_{k,h}$ and $\Pi^{1}_{k,h}$ are the global $L^2$ and $H^1$-projections onto $\Poly_k(\Th)$ (discontinuous piecewise polynomials of degree $\leq k$), respectively, and are understood in terms of their local counterparts. The symmetric positive definite stabilization term is denoted by $S^D(\cdot, \cdot)$. Finally, we post-process $\bs{u}_h \in [\Poly_k(\Th)]^2$ using the Hodge decomposition as follows:
\begin{align}\label{uh.simplyConnected}
    \bs{u}_h = \curl \Pi^{1}_{k,h} \phi_h.
\end{align}

We choose the boundary version of the classical \texttt{Dofi-Dofi} definition of the stabilization term (see \cite[Section 4.2]{brenner.sung:2018:virtual} and \cite{beirao-da-veiga.brezzi.ea:2013:basic}),
\begin{align}\label{def:bd.stab}
    S^D(w_h, v_h) = \sum_{i=1}^{N_{\partial D}^{\texttt{dof}}} \chi_i(w_h) \; \chi_i(v_h) \quad \forall w_h, v_h \in V_h(D),
\end{align}
where operator $\chi_i(\cdot)$ associates the function with its $i$-th degree of freedom, and $N_{\partial D}^{\texttt{dof}}$ is the number of boundary degrees of freedom associated with the element in $D \in \Th$. We have the following property associated with this choice of stabilization (see Remark 4.3 \cite{brenner.sung:2018:virtual}),
\begin{align}
    % \sum_{i=1}^{N_{\partial D}^{\texttt{dof}}} \chi_i(q)^2 \approx \norm{L^\infty(\partial D)}{q}^2 \quad \forall q \in \Poly_k(\partial D), \label{eq:stab.p1}\\
    \sum_{i=1}^{N_{\partial D}^{\texttt{dof}}} \chi_i(w)^2 \lesssim \norm{L^\infty(D)}{w}^2 \quad \forall w \in C(\bar{D}) \label{eq:stab.p2},
\end{align}
where the hidden constants depend on $\Theta$ and $k$.


\subsection{$\Omega$ is multiply connected ($\gamma > 0$).} The $\Poly_k$ Virtual Element Method for approximating the sequence of problems \eqref{eq:P.zeta.chi}-\eqref{eq:P.cj} is as follows.

\begin{enumerate}
    \item Find $(\zeta_h, \xi_h) \in V_h \times V_h^0$ given by
     \begin{align}\label{eq:Ph.zeta.chi}
        (\zeta_h, \xi_h) = (\zeta_{0,h}, \xi_{0,h}) - \frac{(1,\xi_{0,h})}{(1, \xi_{1,h})} (\zeta_{1,h}, \xi_{1,h}),
     \end{align}
     where $(\zeta_{0,h}, \xi_{0,h}), (\zeta_{1,h}, \xi_{1,h}) \in V_h \times V_h^0$ solve the following two coupled systems:
     \begin{align}
            \A_h((\zeta_{0,h},\xi_{0,h}),(\psi_h, \eta_h)) + (\Pi^0_{k,h} \zeta_{0,h}, 1) (\Pi^0_{k,h} \psi_h,1) \nonumber\\
              = \gamma^{-\frac{1}{2}}(\bs{f}, \curl{ \Pi^{1}_{k,h} \psi}) \quad \forall (\psi_h, \eta_h) \in V_h \times V_h^0, \label{eq:Ph.zeta0.xi0}\\
            \A_h((\zeta_{1,h},\xi_{1,h}),(\psi_h, \eta_h)) + (\Pi^0_{k,h} \zeta_{1,h}, 1) (\Pi^0_{k,h} \psi_h,1) \nonumber\\
            = (1,\Pi^0_{k,h} \eta_h) \quad \forall (\psi_h, \eta_h) \in V_h \times V_h^0. \label{eq:Ph.zeta1.xi1}
      \end{align}
      The global coupled bilinear form $\A_h(\cdot, \cdot)$ is defined by
      \begin{align}\label{eq:Ah}
          \A_h((\zeta_h,\xi_h),(\psi_h,\eta_h)) &= a_h(\zeta_h, \psi_h) + \gamma^{\frac{1}{2}} (\Pi^0_{k,h} \psi_h, \Pi^0_{k,h} \xi_h) - \gamma^{\frac{1}{2}} (\Pi^0_{k,h} \zeta_h, \Pi^0_{k,h} \eta_h) \nonumber\\
          &\quad+ a_h(\xi_h, \eta_h) + \beta(\Pi^0_{k,h} \xi_h, \Pi^0_{k,h} \eta_h).
      \end{align}

    \item Find $\phi_h \in V_h$ such that \eqref{eq:Ph.phih} holds.

    \item The coefficients $c_{j,h} \; (1 \leq j \leq m)$, are determined by solving
          \begin{align}\label{eq:P.cj}
            \sum_{j=1}^m a_h(\varphi_{i,h}, \varphi_{j,h}) c_{j,h} = \gamma^{-1} (\bs{f}, \grad \Pi^{1}_{k,h} \varphi_{i,h}) \quad \mbox{for} \quad 1 \leq i \leq m.
          \end{align}
          Where the discrete harmonic functions $\varphi_{j,h}$ are determined by approximating \eqref{eq:P.varphi} as follows:
          \begin{subequations}\label{eq:Ph.varphi}
              \begin{align}
                  a_h(\varphi_{j,h}, v_h) &= 0 \quad \forall v_h \in V_h^0, \label{eq:Ph.varphi.a}\\
                  \varphi_{j,h}|_{\Gamma_0} &= 0, \label{eq:Ph.varphi.b}\\
                  \varphi_{j,h}|_{\Gamma_l} &= \delta_{jl} \quad \mbox{for} \quad 1 \leq l \leq m,\label{eq:Ph.varphi.c}
              \end{align}
          \end{subequations}
          with $a_h(\cdot,\cdot)$ given by \eqref{eq:ah}.
\end{enumerate}

Finally we post-process $\bs{u}_h \in [\Poly_k(\Th)]^2$ using the Hodge decomposition as follows:
\begin{align}\label{uh.notSimplyConnected}
    \bs{u}_h = \curl \Pi^{1}_{k,h} \phi_h + \sum_{j=1}^m c_{j,h} \grad \Pi^{1}_{k,h} \varphi_{j,h}.
\end{align}




	% Convergence analysis and proofs
	\section{Convergence analysis}\label{sec:CA} 

We start by collecting some mathematical tools which will be helpful in the forthcoming analysis. 

\begin{lemma}[Sobolev inequality]\label{lem:Sobolev.inequality} Given any $\delta > 0$,
    \begin{align}\label{eq:Sobolev.inequality}
        \norm{L^\infty(D)}{v} \lesssim h_D^{-1} \norm{L^2(D)}{v} + \seminorm{H^1(D)}{v} + h_D^{\delta} \seminorm{H^{1+\delta}(D)}{v} \quad \forall v \in H^{1+\delta}(D).
    \end{align}
\end{lemma}

\begin{lemma}[Bramble-Hilbert estimates]
    Under the mesh regularity Assumption \ref{meshreg_assum.1}, given any $\delta > 0$, there exists a positive constant independent of $h_D$ such that 
    $$\inf_{p \in \Poly_k(D)} \seminorm{H^1(D)}{\lambda - \psi} \lesssim h_D^{\min(\delta,k)} \seminorm{H^{1+\delta}(D)}{\lambda}, \quad \forall \lambda \in H^{1+\delta}(D).$$
\end{lemma}
% \begin{proof}
%     Using arguments detailed in \cite[Lemma 4.3.8]{Brenner.Scott:2008:MTFEM} and then using interpolation theory \cite[Chapter 14]{Brenner.Scott:2008:MTFEM} to extend to fractional order Sobolev spaces.
% \end{proof}
\begin{lemma}[Trace inequality]\label{lem:trace.inequality}
    Let $e$ be an edge of $D \subset \R^2$. Then, for all $v \in H^{1+\delta}(D)$ with given $\delta > 0$, we have
    $$h_D^{2 \delta} \seminorm{H^{1/2+\delta}(e)}{v}^2 \lesssim \seminorm{H^1(D)}{v}^2 + h_D^{2\delta} \seminorm{H^{1+\delta}(D)}{v}^2.$$
\end{lemma}

\begin{lemma}[$H^1$-projector stability and approximation]\label{lem:L2p.properites} Given $\delta > 0$,
    \begin{align}
        \seminorm{H^1(D)}{\Pi^1_{k,D} v} &\leq \seminorm{H^1(D)}{v} \quad \forall v \in H^1(D), \label{eq:stability.H1p} \\
        \seminorm{H^1(D)}{v - \Pi^1_{k,D} v} &\lesssim h_D^{\min(\delta,k)} \seminorm{H^{1+\delta}(D)}{v} \quad \forall v \in H^{1+\delta}(D). \label{eq:approxH1.H1p} 
        % \norm{L^2(D)}{v - \Pi^1_{k,D} v} &\lesssim h_D^{l+1} \seminorm{H^{l+1}(D)}{v} \quad \forall v \in H^{l+1}(D), \quad 0 \leq l \leq k. \label{eq:approxL2.H1p} \\
        % \seminorm{H^1(D)}{v - \Pi^1_{k,D} v} &\lesssim h_D^{l} \seminorm{H^{l+1}(D)}{v} \quad \forall v \in H^{l+1}(D), \quad 1 \leq l \leq k. \label{eq:approxH1.H1p}.
    \end{align}
\end{lemma}

\begin{lemma}[$L^2$-projector stability and approximation]\label{lem:H1p.properties}
    \begin{align}
       &\norm{L^2(D)}{\Pi^0_{k,D} v} \leq \norm{L^2(D)}{v} \quad \forall v \in L^2(D), \quad
        \norm{H^1(D)}{\Pi^0_{k,D} v} \leq \norm{H^1(D)}{v} \quad \forall v \in H^1(D), \label{eq:stability.L2p} \\
        &\norm{L^2(D)}{v - \Pi^0_{k,D} v} \lesssim h_D^{l+1} \seminorm{H^{l+1}(D)}{v} \quad \forall v \in H^{l+1}(D), \quad 0 \leq l \leq k, \label{eq:approxL2.L2p}
    \end{align}
\end{lemma}

Now we recall the virtual element interpolation operator, which takes any sufficiently smooth function and maps it to the virtual element space. For $s > 1$, the global interpolation operator $\Ih{}: H^s(\Omega) \rightarrow V_h$ is the global counterpart of the local interpolation operator $\ID{}: H^s(D) \rightarrow V_h(D)$ for all $D \in \Th$ such that for any $v \in H^s(D)$,
\begin{align}
    \ID{v(p)} &= v(p) \quad \forall p \in \mathcal{N}^{\partial D}, \label{eq:ID.boundary} \\
    \Pi^0_{k-2,D} \ID{v} &= \Pi^0_{k-2,D} v \label{eq:ID.internal}. 
\end{align}
Here $\mathcal{N}^{\partial D}$ is the set of boundary degrees of freedom associated with the local virtual element space. 
% From \eqref{eq:ID.internal}, it follows that for $k \geq 2$, the interpolation $\ID{v}$ preserves the mean value of $v$ on each element $D$, i.e., given $v \in H^s(D) \cap L_0^2(\Omega)$,
% \begin{align}\label{rem:ID.meanvalue}
%     \int_D \ID{v} = \int_D \Pi^0_{k-2,D} (\ID{v}) = \int_D \Pi^0_{k-2,D} v = \int_D v = 0.
% \end{align}
% However, for $k = 1$, $\ID{v}$ is completely determined by boundary degrees of freedom \eqref{eq:ID.boundary} and does not necessarily preserve the mean value of $v$ on $D$.
    

\begin{lemma}[Interpolation operator stability and approximation]\label{lem:ID.properties}
    Given $\delta > 0$, we have
    \begin{align}
        \seminorm{H^1(D)}{\ID{v}} &\lesssim \seminorm{H^1(D)}{v} + h_D^{\delta} \seminorm{H^{1+ \delta}(D)}{v}, \quad \forall v \in H^{1+\delta}(D) \label{eq:ID.stability} \\
        \seminorm{H^1(D)}{v - \ID v} + \seminorm{H^1(D)}{v - \Pi^1_{k,D} \ID v} &\lesssim h_D^{\min(\delta,k)} \seminorm{H^{1+\delta}(D)}{v}, \quad \forall v \in H^{1+\delta}(D)\label{eq:ID.approximation} \\
        \norm{L^2(D)}{v - \ID v} + \norm{L^2(D)}{v - \Pi^1_{k,D} \ID v} &\lesssim h_D^{{\min(\delta,k)}+1} \seminorm{H^{1+\delta}(D)}{v}, \quad \forall v \in H^{1+\delta}(D)\label{eq:ID.approximation.L2}
    \end{align}   
\end{lemma}

% We will also use the following inverse inequality for virtual element functions \cite[Lemma 2.19]{brenner.guan.ea:2017:some}.
% \begin{lemma}[Inverse inequality]\label{lem:inverse.inequality}
%     Under the mesh regularity assumptions \eqref{meshreg_assum.1}-\eqref{meshreg_assum.2} we have for all $v_h \in V_h(D)$,
%     \begin{align*}
%         \seminorm{H^1(D)}{v_h} \lesssim h_D^{-1} \tnorm{k,D}{v_h},
%     \end{align*}
%     where, $\tnorm{k,D}{\cdot}$ plays the role of $L^2$-norm and is defined as
%     $$\tnorm{k,D}{v_h}^2 = h_D \sum_{e \in \E_D} \norm{L^2(e)}{\Pi^0_{k,e} v_h}^2 + \norm{L^2(D)}{\Pi^0_{k-2,D} v_h}^2.$$
% \end{lemma}

In the subsequent analysis, we work with the following assumption which is made concrete in Remark \ref{coercivity.assum}.
\begin{assumption}\label{assum:coercivity}
    Let $\norm{h_\beta}{\cdot} = \sqrt{a_h(\cdot,\cdot) + \beta (\Pi^0_{k,h} \cdot, \Pi^0_{k,h} \cdot)}$ denote the mesh-dependent energy norm such that 
    \begin{align}
        \seminorm{H^1(\Omega)}{v_h} &\lesssim \norm{h_\beta}{v_h} \quad \forall v_h \in V_h^0, \label{eq:coercivity.Dirichlet} \\
        \seminorm{H^1(\Omega)}{v_h} &\lesssim \norm{h_\beta}{v_h} \quad \forall v_h \in V_h \quad \mbox{with} \quad (v_h,1) = 0. \label{eq:coercivity.Neumann} 
    \end{align}
\end{assumption}

We also define the piecewise-broken $H^1$-seminorm by
\begin{align*}
    \seminorm{1,h}{v} := \left( \sum_{D \in \Th} \seminorm{H^1(D)}{v}^2 \right)^{\frac{1}{2}} \quad \forall v \in H^1(\Th).
\end{align*}

Some useful estimates in the mesh-dependent norm and piecewise broken semi-norm are in order.
\begin{lemma}\label{lem:est.mesh.dependent}
    The following estimates hold,
    \begin{align}
        \norm{h_0}{\Ih{v} - \Pi^1_{k,h} \Ih{v}} + \seminorm{1,h}{\Ih{v} - \Pi^1_{k,h} \Ih{v}} &\lesssim h^{\min(\delta,k)} \seminorm{H^{1+\delta}(\Omega)}{v} \quad \forall v \in H^{1+\delta}(\Omega), \label{eq:est.Ih.h.norm} \\
        \norm{h_\beta}{v- \Ih{v}} + \norm{h_\beta}{v- \Pi^1_{k,h} v} &\lesssim h^{\min(\delta,k)} \seminorm{H^{1+\delta}(\Omega)}{v} \quad \forall v \in H^{1+\delta}(\Omega). \label{eq:est.hbeta.norm}
    \end{align}
\end{lemma}

\begin{remark}\label{coercivity.assum}
    \corr{}{}{[JT: To do - Making Assumption 5.7 concrete, for shape-regular meshes vs meshes with small edges]}
    
\end{remark}



\subsection{$\Omega$ is simply connected $(\gamma = 0)$} Our first aim is to estimate the error $\xi - \xi_h$. To this end, we first write down the error equation by testing \eqref{eq:P.rho} with $\psi_h \in V_h$ and subtracting it from \eqref{eq:Ph.rhoh},
\begin{align}\label{eq:erreq.rho}
    (\curl{\rho}, \curl{\psi_h}) - a_h(\rho_h,\psi_h) = (\bs{f}, \curl{(\psi_h - \Pi^1_{k,h}(\psi_h))}) \quad \forall \psi_h \in V_h.
\end{align}
We note that to obtain the above error equation we also used the definition of $\Pi^0_{k,h}$ and the fact that $\rho, \rho_h \in L_0^2(\Omega)$. Furthermore, in view of \eqref{eq:P.rho}, \eqref{eq:Ph.rhoh}, $\rho, \rho_h \in L_0^2(\Omega)$ and Assumption \ref{assum:coercivity}, we have the following relations:
\begin{align}\label{eq:curl.data}
    \norm{L^2(\Omega)}{\curl{\rho}} \leq \norm{L^2(\Omega)}{\bs{f}}, \quad \mbox{and} \quad \norm{L^2(\Omega)}{\curl{\rho_h}} \lesssim \norm{L^2(\Omega)}{\bs{f}}.
\end{align}

In the following lemma we estimate the error for $\rho_h$ in the dual norm using a duality argument.

\begin{lemma}\label{lem:rhoh.dual}
    For any $\epsilon > 0$, there exists a positive constant dependent on $\epsilon$ and independent of $h$ such that
    \begin{align}\label{eq:err.est.rho}
       | (\rho-\rho_h, \chi) | \lesssim h^{\min{((\pi / \omega) - \epsilon, k)}} \norm{H^1(\Omega)}{\chi} \norm{L^2(\Omega)}{\bs{f}} \quad \forall \chi \in H^1(\Omega).
    \end{align}

\end{lemma}

\begin{proof}
    Given arbitary $\chi \in H^1(\Omega)$, let $\lambda \in H^1(\Omega)$ solve the following dual problem:
\begin{align}\label{eq:dual.rho}
    (\curl{\psi}, \curl{\lambda}) + (\psi,1) (\lambda,1) = (\psi, \chi) \quad \forall \psi \in H^1(\Omega).
\end{align}
Testing \eqref{eq:dual.rho} with $\rho - \rho_h \in H^1(\Omega)$ and exploiting $\rho, \rho_h \in L_0^2(\Omega)$, we get
\begin{align}\label{eq:err.rho.1}
    (\rho - \rho_h,\chi) &= (\curl{(\rho - \rho_h)}, \curl{\lambda}) + (\rho - \rho_h,1) (\lambda,1) \nonumber\\
    &= (\curl{(\rho - \rho_h)}, \curl{(\lambda- \Ih{\lambda})} ) + (\curl{(\rho - \rho_h)}, \curl{\Ih{\lambda}}).
\end{align}
The first term in \eqref{eq:err.rho.1} is bounded using the Cauchy-Schwarz inequality and \eqref{eq:curl.data}, to get
\begin{align}\label{eq:err.rho.1.a}
    (\curl{(\rho - \rho_h)}, \curl{(\lambda- \Ih{\lambda})} ) &\leq \norm{L^2(\Omega)}{\curl{(\rho - \rho_h)}} \norm{L^2(\Omega)}{\curl{(\lambda - \Ih{\lambda})}}. \nonumber \\
    &\lesssim \norm{L^2(\Omega)}{\bs{f}} \seminorm{H^1(\Omega)}{\lambda - \Ih{\lambda}}.
\end{align}

The second term in \eqref{eq:err.rho.1} is bounded using the error equation \eqref{eq:erreq.rho} for $\psi_h = \Ih{\lambda}$ as follows:
\begin{align}\label{eq:err.rho.1.b}
    (\curl{(\rho - \rho_h)}, \curl{\Ih{\lambda}}) &= (\curl{\rho}, \curl{\Ih{\lambda}}) - (\curl{\rho_h}, \curl{\Ih{\lambda}}) \nonumber\\
    &= (\bs{f}, \curl{(\Ih{\lambda} - \Pi^1_{k,h} \Ih{\lambda})}) + a_h(\rho_h,\Ih{\lambda}) - (\curl{\rho_h}, \curl{\Ih{\lambda}}) \nonumber\\
    &\leq \norm{L^2(\Omega)}{\bs{f}} \seminorm{1,h}{(I-\Pi^1_{k,h})\Ih{\lambda}} + a_h(\rho_h,\Ih{\lambda}) - (\curl{\rho_h}, \curl{\Ih{\lambda}}).
\end{align}

The difference in \eqref{eq:err.rho.1.b} can be rewritten as follows:
\begin{align}
    &a_h(\rho_h,\Ih{\lambda}) - (\curl{\rho_h}, \curl{\Ih{\lambda}})  \nonumber\\
    &\quad=\sum_{D \in \Th} (\curl{\Pi^1_{k,D} \rho_h, \curl{\Pi^1_D {\ID{\lambda}}}}) + S^D((I - \Pi^1_{k,D})\rho_h, (I - \Pi^1_{k,D})\ID{\lambda}) -\nonumber\\
    &\qquad\qquad(\curl{\rho_h}, \curl{(\Ih{\lambda}-\Pi^1_{k,D}\ID{\lambda})}) - (\curl{\rho_h}, \curl{\Pi^1_{k,D}\ID{\lambda}}) \nonumber\\
    &\quad= \sum_{D \in \Th} S^D((I - \Pi^1_{k,D})\rho_h, (I - \Pi^1_{k,D})\ID{\lambda}) - (\curl{\rho_h}, \curl{(\ID{\lambda} - \Pi^1_{k,D}\ID{\lambda})}) \nonumber\\
    &\quad\leq  (\sum_{D \in \Th} S^D((I - \Pi^1_{k,D})\rho_h, (I - \Pi^1_{k,D})\rho_h))^{1/2} (\sum_{D \in \Th} S^D((I - \Pi^1_{k,D})\ID{\lambda}, (I - \Pi^1_{k,D})\ID{\lambda}))^{1/2} \nonumber\\
    &\qquad\qquad (\sum_{D \in \Th} \seminorm{H^1(D)}{\rho_h}^2)^{1/2} (\sum_{D \in \Th} \seminorm{H^1(D)}{\ID{\lambda} - \Pi^1_{k,D}\ID{\lambda}}^2)^{1/2} \nonumber\\
    &\quad\lesssim \norm{h_0}{(I - \Pi^1_{k,h})\rho_h} \norm{h_0}{(I-\Pi^1_{k,h})\Ih{\lambda}} + \seminorm{H^1(\Omega)}{\rho_h} \seminorm{1,h}{(I - \Pi^1_{k,h})\Ih{\lambda}}. \label{eq:err.rho.1.c}
\end{align}
Where in the passage to the second equality we used the definition of $\Pi^1_{k,D}$ projector followed by Cauchy-Schwarz inequality and finally the definition of $\norm{h}{\cdot}$ and $\seminorm{1,h}{\cdot}$. Using the linearity and idempotency of $\Pi^1_{k,D}$ and $\rho_h \in L_0^2(\Omega)$ we have that
\begin{align*}
    \norm{h_0}{(I - \Pi^1_{k,h} \rho_h)}^2 &= S((I - \Pi^1_{k,h})\rho_h, (I - \Pi^1_{k,h})\rho_h) \nonumber\\ 
    &\overset{\eqref{eq:Ph.rhoh}}{=} (\bs{f}, \curl{\Pi^1_{k,h} \rho_h}) - (\curl{\Pi^1_{k,h} \rho_h}, \curl{\Pi^1_{k,h} \rho_h}) \nonumber\\
    &\leq \sum_{D \in \Th} \norm{L^2(D)}{\bs{f}} \norm{H^1(D)}{\Pi^1_{k,D} \rho_h} + \norm{H^1(D)}{\Pi^1_{k,D} \rho_h}^2 \nonumber\\
    &\overset{\eqref{eq:stability.H1p}}{\leq}  \sum_{D \in \Th} \norm{L^2(D)}{\bs{f}} \norm{H^1(D)}{\rho_h} + \norm{H^1(D)}{\rho_h}^2 \overset{\eqref{eq:curl.data}}{\lesssim} \norm{L^2(\Omega)}{\bs{f}}^2.
\end{align*}
The regularity of the dual problem from Lemma \ref{lem:reg.pure-neumann}, the above bound, the estimate \eqref{eq:est.Ih.h.norm} with $\delta = (\pi/\omega)-\epsilon$, and \eqref{eq:curl.data} lead to the following bound on \eqref{eq:err.rho.1.c},
\begin{align*}
    a_h(\rho_h,\Ih{\lambda}) - (\curl{\rho_h}, \curl{\Ih{\lambda}}) \lesssim h^{\min{((\pi / \omega)-\epsilon,k)}} \norm{L^2(\Omega)}{\bs{f}} \seminorm{H^{1+(\pi/\omega)-\epsilon}(\Omega)}{\lambda}.  
\end{align*}

Substituting in \eqref{eq:err.rho.1.b}, again using \eqref{eq:est.Ih.h.norm} and the data dependence relation of the dual problem in Lemma \ref{lem:reg.pure-neumann} yield
\begin{align*}
    (\curl{(\rho - \rho_h)}, \curl{\Ih{\lambda}}) &\lesssim  h^{\min{((\pi / \omega)-\epsilon,k)}} \norm{L^2(\Omega)}{\bs{f}} \norm{H^{1}(\Omega)}{\chi}.
\end{align*}
Substituting the above bound and \eqref{eq:err.rho.1.a} into \eqref{eq:err.rho.1}, and using the definition of piecewise-broken $H^1$-seminorm and \eqref{eq:ID.approximation} yields the desired error estimate \eqref{eq:err.est.rho}.
\end{proof}


It remains to estimate the error for $\xi_h$. 
\begin{lemma}\label{lem:xi.est}
    For any $\epsilon > 0$, there exists a positive constant dependent on $\epsilon$ and independent of $h$ such that for $k = 1,2$,
    \begin{align}
        \seminorm{H^1(\Omega)}{\xi - \xi_h} &\lesssim h^{\min{((\pi / \omega) - \epsilon, k)}} \norm{L^2(\Omega)}{\bs{f}}, \label{eq:err.est.xi} \\
         \seminorm{H^1(\Omega)}{\xi - \Pi^1_{k,h} \xi_h}&\lesssim h^{\min{((\pi/\omega) - \epsilon, k)}} \norm{L^2(\Omega)}{\bs{f}}. \label{eq:err.est.Proj_xi}
    \end{align}
\end{lemma}
\begin{proof}
    Given $\rho_h \in V_h \cap L_0^2(\Omega)$, let $\tilde{\xi}_0 \in H_0^1(\Omega)$ solve the following auxiliary problem,
    \begin{align}\label{eq:P.aux}
        (\curl{\tilde{\xi}_0}, \curl{\eta}) + \beta(\tilde{\xi}_0, \eta) = (\rho_h, \eta) \quad \forall \eta \in H_0^1(\Omega).
    \end{align}
    Since $\rho_h \in V_h \subset H^1(\Omega)$, due to Lemma \ref{lem:reg.reac-diff}, 
    \begin{align}\label{tildexi0.reg}
        \tilde \xi_0 \in H^{1+(\pi/\omega)-\epsilon}(\Omega) \quad \mbox{and} \quad \norm{H^{1+(\pi/\omega)-\epsilon}(\Omega)}{\tilde \xi_0} \lesssim \norm{H^1(\Omega)}{\rho_h}.
    \end{align}
    It follows now from \eqref{eq:P.xi0}, \eqref{eq:P.aux}, and the duality estimate \eqref{eq:err.est.rho} that
    \begin{align}\label{eq:xi0.a}
        \seminorm{H^1(\Omega)}{\xi_0 - \tilde{\xi}_0} &\lesssim h^{\min{((\pi/\omega) - \epsilon, k)}} \norm{L^2(\Omega)}{\bs{f}}.
    \end{align}

It follows from \eqref{eq:P.aux} and \eqref{eq:Ph.xi0h} that $\xi_{0,h}$ is also a virtual element approximation of $\tilde \xi_0$ and hence we have the following analogous abstract error estimate following \cite[Section 4.3]{Brenner-Sung-2018-VEM-Small-Edges-Faces},
\begin{align}\label{err.abs.xi0h}
    \norm{h_{\beta}}{\tilde \xi_0 - \xi_{0,h}} &\lesssim \norm{h_\beta}{\tilde \xi_0 - \Ih{\tilde \xi_0}} + \seminorm{1,h}{\tilde \xi_0 - \Pi^1_{k,h} \tilde \xi_0} + \norm{h_\beta}{\tilde \xi_0 - \Pi^1_{k,h} \tilde \xi_0} \nonumber\\
    &\qquad + \beta h \norm{L^2(\Omega)}{\tilde \xi_0 - \Pi^0_{k,h} \tilde \xi_0} + \sup_{w_h \in V_h^0 \backslash \{0\}} \frac{(\rho_h, w_h - \Pi^0_{k,h} w_h)}{\seminorm{H^1(\Omega)}{w_h}}.
\end{align}

The numerator in the last term of \eqref{err.abs.xi0h} is estimated using the definition of $\Pi^0_{k,h}$, the fact that it minimizes it's respective norm, the fact $\rho_h, w_h \in V_h \subset H^1(\Omega)$ and the approximation property of $\Pi^0_{k,h}$ follows:
\begin{align}
    (\rho_h, w_h - \Pi^0_{k,h} w_h) &= (\rho_h - \Pi^0_{0,h} \rho_h, w_h - \Pi^0_{k,h} w_h) \nonumber\\
    &\leq \norm{L^2(\Omega)}{\rho_h - \Pi^0_{0,h} \rho_h} \norm{L^2(\Omega)}{w_h - \Pi^0_{0,h} w_h} \nonumber\\
    &\lesssim h^2 \seminorm{H^1(\Omega)}{\rho_h} \seminorm{H^1(\Omega)}{w_h}. \label{eq:err.abs.xi0h.source}
\end{align}

Using the approximation properties stated in \eqref{eq:est.hbeta.norm}, the definition of $\seminorm{1,h}{\cdot}$, the estimates \eqref{eq:approxH1.H1p} and \eqref{eq:approxL2.L2p} for $\delta = (\pi/\omega) - \epsilon$, and \eqref{eq:err.abs.xi0h.source}, we can get the following concrete bound on \eqref{err.abs.xi0h}.
\begin{align*}
    \norm{h_{\beta}}{\tilde \xi_0 - \xi_{0,h}} &\lesssim h^{\min{((\pi/\omega) - \epsilon, k)}} (\seminorm{H^{1 + (\pi/\omega) - \epsilon}(\Omega)}{\tilde \xi_0} + \seminorm{H^1(\Omega)}{\rho_h}) \nonumber\\
    &\hspace{-0.65cm}\overset{\eqref{tildexi0.reg},\eqref{eq:curl.data}}{\lesssim} h^{\min{((\pi/\omega) - \epsilon, k)}} \norm{L^2(\Omega)}{\bs{f}}.
\end{align*}
Now, using Assumption \ref{assum:coercivity} in the above estimate yields
\begin{align}\label{eq:xi0h.b}
    \seminorm{H^1(\Omega)}{\tilde \xi_0 - \xi_{0,h}} \lesssim h^{\min{((\pi/\omega) - \epsilon, k)}} \norm{L^2(\Omega)}{\bs{f}}.
\end{align}
Finally, a triangle inequality alongwith \eqref{eq:xi0.a} and \eqref{eq:xi0h.b} leads to \eqref{eq:err.xi0}
\begin{align}\label{eq:err.xi0}
     \seminorm{H^1(\Omega)}{\xi_0 - \xi_{0,h}} \lesssim h^{\min{((\pi/\omega) - \epsilon, k)}} \norm{L^2(\Omega)}{\bs{f}}.
\end{align}
Similarly, we can get the following control on $\xi_1 - \xi_{1,h}$ (see \eqref{eq:P.xi1} and \eqref{eq:Ph.xi1h})
\begin{align}\label{eq:err.xi1}
     \seminorm{H^1(\Omega)}{\xi_1 - \xi_{1,h}} \lesssim h^{\min{((\pi/\omega) - \epsilon, k)}}.
\end{align}
The estimate \eqref{eq:err.est.xi} now follows from \eqref{eq:err.xi0}, \eqref{eq:err.xi1}, \eqref{eq:P.xi}, and \eqref{eq:Ph.xih}. Since, $\xi \in H^{1+(\pi/\omega) - \epsilon}(\Omega)$ we can get the estimate \eqref{eq:err.est.Proj_xi} using Lemma \ref{lem:H1p.properties} and \eqref{eq:err.est.xi}  as follows,
\begin{align*}
    \seminorm{1,h}{\xi - \Pi^1_{k,h} \xi_h} &\leq \seminorm{1,h}{\xi - \Pi^1_{k,h} \xi} + \seminorm{1,h}{\Pi^1_{k,h} (\xi - \xi_h)} \nonumber\\
    &\lesssim h^{\min{((\pi/\omega) - \epsilon, k)}} \norm{H^{1+(\pi/\omega) - \epsilon}(\Omega)}{\xi} + \seminorm{H^1(\Omega)}{\xi - \xi_h} \nonumber\\
    &\lesssim h^{\min{((\pi/\omega) - \epsilon, k)}} \norm{L^2(\Omega)}{\bs{f}}.
\end{align*}
In the last step we used Lemma \ref{lem:reg.reac-diff} on \eqref{eq:xi} followed by \eqref{eq:curl.data}. This completes the proof.
\end{proof}

The error analysis for $\phi_h$ is analogous to that of $\xi_h$.

\begin{lemma}\label{lem:est.phi}
    For any $\epsilon > 0$, there esists a positive constant dependent on $\epsilon$ and independent of $h$ such that for $k = 1,2$,
    \begin{align}
        \seminorm{H^1(\Omega)}{\phi - \phi_h} &\lesssim h^{\min{((\pi / \omega) - \epsilon, k)}} \left(\norm{H^1(\Omega)}{\xi_h} + \norm{L^2(\Omega)}{\bs{f}}\right), \label{eq:err.est.phi} \\
        \seminorm{1,h}{\phi - \Pi^1_{k,h} \phi_h} &\lesssim h^{\min{((\pi/\omega) - \epsilon, k)}} \left( \norm{H^1(\Omega)}{\xi} + \norm{H^1(\Omega)}{\xi_h} +\norm{L^2(\Omega)}{\bs{f}}\right). \label{eq:err.est.Proj_phi}
    \end{align}
\end{lemma}
\begin{proof}
    Given $\xi_h \in V_h \cap L_0^2(\Omega)$, let $\tilde{\phi} \in H^1(\Omega)$ solve the following auxiliary problem,
    \begin{align}\label{eq:P.aux.phi}
        (\curl{\tilde \phi}, \curl{\psi}) + (\tilde \phi, 1) (\tilde \psi, 1) = (\xi_h, \psi) \quad \forall \psi \in H^1(\Omega).
    \end{align}
    Since $\xi_h \in V_h \subset H^1(\Omega)$, due to Lemma \ref{lem:reg.pure-neumann}, 
    \begin{align}\label{tildephi.reg}
        \tilde \phi \in H^{1+(\pi/\omega)-\epsilon}(\Omega) \quad \mbox{and} \quad \norm{H^{1+(\pi/\omega)-\epsilon}(\Omega)}{\tilde \phi} \lesssim \norm{H^1(\Omega)}{\xi_h}.
    \end{align}
    Comparing \eqref{eq:P.phi} and \eqref{eq:P.aux.phi}, using a standard Poincar\'e inequality for $\xi - \xi_h \in H_0^1(\Omega)$ alongwith \eqref{eq:err.est.xi} gives,
    \begin{align}\label{eq:phi.a}
        \seminorm{H^1(\Omega)}{\phi - \tilde{\phi}} \lesssim \norm{L^2(\Omega)}{\xi - \xi_h} \lesssim h^{\min{((\pi/\omega) - \epsilon, k)}} \norm{L^2(\Omega)}{\bs{f}}.
    \end{align}

    It follows from \eqref{eq:P.aux.phi} and \eqref{eq:Ph.phih} that $\phi_h$ is also a virtual element approximation of $\tilde \phi$ and hence using the fact the $\rho, \rho_h \in L_0^2(\Omega)$ we have the following analogous abstract error estimate following \cite[Section 4.3]{Brenner-Sung-2018-VEM-Small-Edges-Faces},
    \begin{align*}
        \norm{h_{\beta}}{\tilde \phi - \phi_h} &\lesssim \norm{h_0}{\tilde \phi - \Ih{\tilde \phi}} + \seminorm{1,h}{\tilde \phi - \Pi^1_{k,h} \tilde \phi} + \norm{h_0}{\tilde \phi - \Pi^1_{k,h} \tilde \phi} + \sup_{w_h \in V_h^0 \backslash \{0\}} \frac{(\xi_h, w_h - \Pi^0_{k,h} w_h)}{\seminorm{H^1(\Omega)}{w_h}}.
    \end{align*}
    Following similar arguments as in the proof of Lemma \ref{lem:xi.est} and \eqref{tildephi.reg}, we can get the following concrete bound,
    \begin{align}\label{eq:phi.b}
        \seminorm{H^1(\Omega)}{\tilde \phi - \phi_h} &\lesssim h^{\min{((\pi/\omega) - \epsilon, k)}} (\norm{H^{1 + (\pi/\omega) - \epsilon}(\Omega)}{\tilde \phi} + \seminorm{H^1(\Omega)}{\xi_h}) \nonumber\\
        &\lesssim h^{\min{((\pi/\omega) - \epsilon, k)}} \norm{H^1(\Omega)}{\xi_h}.
    \end{align}
    Combining \eqref{eq:phi.a} and \eqref{eq:phi.b} using a triangle inequality leads to \eqref{eq:err.est.phi}. 

    Moreover, it follows from \eqref{eq:P.phi} and Lemma \ref{lem:reg.pure-neumann} that 
    \begin{align}\label{eq:reg.phi}
        \phi \in H^{1+(\pi/\omega) - \epsilon}(\Omega) \quad \mbox{and} \quad \norm{H^{1+(\pi/\omega) - \epsilon}(\Omega)}{\phi} \lesssim \norm{H^1(\Omega)}{\xi}.    
    \end{align}
    Hence, using similar arguments as in the proof of \eqref{eq:err.est.Proj_xi} with \eqref{eq:reg.phi} leads to \eqref{eq:err.est.Proj_phi}.
\end{proof}

In view of Lemmas \ref{lem:xi.est}, \ref{lem:est.phi}, \eqref{u.simplyConnected} and \eqref{uh.simplyConnected}, we readily have the following result for simply connected domains.
\begin{theorem}\label{thm:simplyconnected}
    The approximations $\xi_h$ and $\bs{u}_h$ obtained by $\Poly_k$-virtual finite element method satisfy 
    \begin{align}\label{eq:err.est.simplyconnected}
        \norm{L^2(\Omega)}{\bs{u} - \bs{u}_h} + \seminorm{H^1(\Omega)}{\curl{\bs{u}} - \xi_h} \lesssim h^{\min{((\pi / \omega) - \epsilon, k)}} \left( \norm{H^1(\Omega)}{\xi} + \norm{H^1(\Omega)}{\xi_h} +\norm{L^2(\Omega)}{\bs{f}}\right).
    \end{align}
    for any $\epsilon > 0$, and $k = 1,2$, where $\omega$ is the largest angle at the corners of $\Omega$.
\end{theorem} 
\begin{proof}
    The estimate for $\bs{u} - \bs{u}_h$ follows from \eqref{eq:err.est.Proj_phi} on observing the following,
    $$\norm{L^2(\Omega)}{\bs{u} - \bs{u}_h} = \norm{L^2(\Omega)}{\curl{\phi} - \curl{\Pi^1_{k,h} \phi_h}}=\seminorm{1,h}{\phi - \Pi^1_{k,h} \phi_h}.$$
    Recalling that $\xi = \curl{\bs{u}}$, the estimate for $\curl{\bs{u}} - \xi_h$ follows from \eqref{eq:err.est.xi}.
\end{proof}


\subsection{$\Omega$ is multiply connected $(\gamma > 0)$}
The convergence analysis for the multiply connected case follows ideas similar to those in the simply connected case extended to the coupled setting. We first write down the error equation for $(\zeta_{0,h}, \xi_{0,h})$ by testing \eqref{eq:P.zeta0.xi0} with $(\psi_h, \eta_h) \in V_h \times V_h^0$, subtracting it from \eqref{eq:Ph.zeta0.xi0} and using $\zeta_0, \zeta_{0,h} \in L_0^2(\Omega)$,
\begin{align}\label{eq:err.zeta0.xi0}
    \A((\zeta_0,\xi_0),(\psi_h,\eta_h)) - \A_h((\zeta_{0,h},\xi_{0,h}),(\psi_h,\eta_h)) = \gamma^{-1/2} (\bs{f}, \curl{(\psi_h - \Pi^1_{k,h} \psi_h)})
\end{align}

Furthermore, testing \eqref{eq:P.zeta0.xi0} with $(\zeta_0, \xi_0)$ and Young's inequality gives

\begin{align}
    \seminorm{H^1(\Omega)}{\zeta_0}^2 + (\zeta_0,1)^2 + \seminorm{H^1(\Omega)}{\xi_0}^2 &\lesssim \norm{L^2(\Omega)}{\bs{f}}^2, \nonumber\\
    \norm{H^1(\Omega)}{\zeta_0}^2 + \seminorm{H^1(\Omega)}{\xi_0}^2 &\lesssim \norm{L^2(\Omega)}{\bs{f}}^2. \label{eq:data.zeta0.xi0}
\end{align}
where in the last inequality is a consequence of Poincar\'e-Wirtinger inequality. Similarly, testing \eqref{eq:Ph.zeta0.xi0} with $(\zeta_{0,h}, \xi_{0,h})$, and in view of Assumption \ref{assum:coercivity} and stability of $\Pi^1_{k,h}$, we obtain
\begin{align}\label{eq:data.zeta0h.xi0h}
    \norm{H^1(\Omega)}{\zeta_{0,h}}^2 + \seminorm{H^1(\Omega)}{\xi_{0,h}}^2 \lesssim \norm{L^2(\Omega)}{\bs{f}}.
\end{align} 
% Furthermore, testing \eqref{eq:P.zeta0.xi0} with $(\zeta_0, \xi_0)$ and \eqref{eq:Ph.zeta0.xi0} with $(\zeta_{0,h}, \xi_{0,h})$, and in view of $\zeta_0, \zeta_{0,h} \in L_0^2(\Omega)$, Assumption \ref{assum:coercivity} and Young's inequality, we can obtain

% \begin{align}\label{eq:data.zeta0.xi0}
%     \seminorm{H^1(\Omega)}{\zeta_0}^2 + \seminorm{H^1(\Omega)}{\xi_0}^2 \lesssim \norm{L^2(\Omega)}{\bs{f}}^2, \quad \mbox{and} \quad
%     \seminorm{H^1(\Omega)}{\zeta_{0,h}}^2 + \seminorm{H^1(\Omega)}{\xi_{0,h}}^2 \lesssim \norm{L^2(\Omega)}{\bs{f}}.
% \end{align}

We estimate the error for $\xi_{0,h}$ in $H^1(\Omega)'$ by using a duality argument. To this end, consider the following dual problem. Given $\chi \in H^1(\Omega)$, let $(\lambda, \mu) \in H^1(\Omega) \times H_0^1(\Omega)$ solve
\begin{align}\label{eq:DP.coupled}
    \A((\psi,\eta),(\lambda,\mu)) + (\psi,1)(\lambda,1) = (\psi,\chi) \quad \forall (\psi,\eta) \in H^1(\Omega) \times H_0^1(\Omega).
\end{align} 
Observe that testing the above well-posed dual problem with $(\lambda, \mu)$ gives the following stability estimate,
\begin{align}\label{eq:DP.stab.est}
    \norm{H^1(\Omega)}{\lambda}^2 + \norm{H^1(\Omega)}{\mu}^2 \lesssim \norm{L^2(\Omega)}{\chi}^2.
\end{align}
Using the definition of $\A(\cdot,\cdot)$, we can rewrite \eqref{eq:DP.coupled} as the following coupled system:
\begin{align}
    (\curl{\psi}, \curl{\lambda}) + (\psi,1)(\lambda,1) &= (\psi,\chi) + \gamma^{1/2}(\psi,\mu) \quad \forall \psi \in H^1(\Omega), \           \label{eq:DP.coupled.1}\\
    (\curl{\eta}, \curl{\mu}) + \beta(\eta, \mu) &= -\gamma^{1/2}(\lambda,\eta) \quad \forall \eta \in H_0^1(\Omega). \label{eq:DP.coupled.2}
\end{align}
Thus, using Lemma \ref{lem:reg.pure-neumann} on \eqref{eq:DP.coupled.1}, and Lemma \ref{lem:reg.reac-diff} on \eqref{eq:DP.coupled.2}, followed by the stability estimate \eqref{eq:DP.stab.est}, we have the following regularity result for the dual problem,
\begin{align}\label{eq:DP.regularity}
    \norm{H^{1+(\pi/\omega)-\epsilon}(\Omega)}{\lambda} + \norm{H^{1+(\pi/\omega)-\epsilon}(\Omega)}{\mu} \lesssim \norm{H^1(\Omega)}{\chi}.
\end{align}

Testing \eqref{eq:DP.coupled} with $(\psi, \eta) = (\zeta_0 - \zeta_{0,h}, \xi_0 - \xi_{0,h}) \in H^1(\Omega) \times H_0^1(\Omega)$ and using the fact that $\xi_0, \xi_{0,h} \in L_0^2(\Omega)$, we have
\begin{align}
    (\xi_0 - \xi_{0,h}, \chi)  &= \A((\zeta_0 - \zeta_{0,h}, \xi_0 - \xi_{0,h}),(\lambda,\mu)) \nonumber\\
    &=  \A((\zeta_0 - \zeta_{0,h}, \xi_0 - \xi_{0,h}),(\lambda - \Ih{\lambda},\mu - \Ih{\mu})) + \A((\zeta_0 - \zeta_{0,h}, \xi_0 - \xi_{0,h}),(\Ih{\lambda},\Ih{\mu})). \label{eq:err.coupled.xi0.1}
\end{align}

The first term in \eqref{eq:err.coupled.xi0.1} is bounded using the definition of $\A$, Cauchy-Schwarz inequality, \eqref{eq:data.zeta0.xi0}, and \eqref{eq:data.zeta0h.xi0h} to get
\begin{align}\label{eq:err.coupled.xi0.1.a}
    &\A((\zeta_0 - \zeta_{0,h}, \xi_0 - \xi_{0,h}),(\lambda - \Ih{\lambda},\mu - \Ih{\mu})) \nonumber\\
    &\lesssim \left(\norm{H^1(\Omega)}{\zeta_0 - \zeta_{0,h}} + 
    \seminorm{H^1(\Omega)}{\xi_0 - \xi_{0,h}}\right)
     \times \left( \norm{H^1(\Omega)}{\lambda - \Ih{\lambda}} + \seminorm{H^1(\Omega)}{\mu - \Ih{\mu}} \right) \nonumber\\
    &\lesssim  \left( \norm{H^1(\Omega)}{\lambda - \Ih{\lambda}} + \seminorm{H^1(\Omega)}{\mu - \Ih{\mu}} \right) \norm{L^2(\Omega)}{\bs{f}}.
\end{align}

The second term in \eqref{eq:err.coupled.xi0.1} is bounded using the error equation \eqref{eq:erreq.rho} for $(\psi_h,\eta_h) = (\Ih{\lambda}, \Ih{\mu})$ as follows:
\begin{align}\label{eq:err.coupled.xi0.1.b}
    &\A((\zeta_0 - \zeta_{0,h}, \xi_0 - \xi_{0,h}),(\Ih{\lambda},\Ih{\mu})) \nonumber\\
    &= \gamma^{-1/2} (\bs{f}, \curl{(\Ih{\lambda} - \Pi^1_{k,h} \Ih{\lambda})}) + \A_h((\zeta_{0,h},\xi_{0,h}),(\Ih{\lambda},\Ih{\mu})) - \A((\zeta_{0,h},\xi_{0,h}),(\Ih{\lambda},\Ih{\mu})) \nonumber\\
    &\lesssim \norm{L^2(\Omega)}{\bs{f}} \seminorm{1,h}{(I-\Pi^1_{k,h})\Ih{\lambda}} + \A_h((\zeta_{0,h},\xi_{0,h}),(\Ih{\lambda},\Ih{\mu})) - \A((\zeta_{0,h},\xi_{0,h}),(\Ih{\lambda},\Ih{\mu})).
\end{align}
The difference in \eqref{eq:err.coupled.xi0.1.b} can be rewritten as follows:
\begin{align*}
    &\A_h((\zeta_{0,h},\xi_{0,h}),(\Ih{\lambda},\Ih{\mu})) - \A((\zeta_{0,h},\xi_{0,h}),(\Ih{\lambda},\Ih{\mu})) \nonumber\\
    &\quad= \left(a_h(\zeta_{0,h}, \Ih{\lambda}) - (\curl{\zeta_{0,h}}, \curl{\Ih{\lambda}})\right) 
    + \left(a_h(\xi_{0,h}, \Ih{\mu}) - (\curl{\xi_{0,h}}, \curl{\Ih{\mu}})\right) \\
    &\qquad +\left(\gamma^{1/2} (\Pi^0_{k,h} \Ih{\lambda}, \Pi^0_{k,h}\xi_{0,h}) - \gamma^{1/2}(\Ih{\lambda}, \xi_{0,h})\right) + \left(\gamma^{1/2} (\xi_{0,h}, \Ih{\mu})-\gamma^{1/2}(\Pi^0_{k,h} \zeta_{0,h}, \Pi^0_{k,h} \Ih{\mu})\right) \\
    &\qquad  +\left(\beta(\Pi^0_{k,h}\xi_{0,h}, \Pi^0_{k,h}\Ih{\mu}) - \beta(\xi_{0,h}, \Ih{\mu})\right) \\
    &\quad =: T_1 + T_2 + T_3 + T_4 + T_5.
\end{align*}







	\section{Proof of some Lemmas}\label{sec:CA.proofs} 
% We choose the boundary version of the classical \texttt{Dofi-Dofi} definition of the stabilization term (see \cite[Section 4.2]{brenner.sung:2018:virtual} and \cite{beirao-da-veiga.brezzi.ea:2013:basic}),
% \begin{align}\label{def:bd.stab}
%     S^D(w_h, v_h) = \sum_{i=1}^{N_{\partial D}^{\texttt{dof}}} \chi_i(w_h) \; \chi_i(v_h) \quad \forall w_h, v_h \in V_h(D),
% \end{align}
% where operator $\chi_i(\cdot)$ associates the function with its $i$-th degree of freedom, and $N_{\partial D}^{\texttt{dof}}$ is the number of boundary degrees of freedom associated with the element in $D \in \Th$. We have the following property associated with this choice of stabilization (see Remark 4.3 \cite{brenner.sung:2018:virtual}),
% \begin{align}
%     % \sum_{i=1}^{N_{\partial D}^{\texttt{dof}}} \chi_i(q)^2 \approx \norm{L^\infty(\partial D)}{q}^2 \quad \forall q \in \Poly_k(\partial D), \label{eq:stab.p1}\\
%     \sum_{i=1}^{N_{\partial D}^{\texttt{dof}}} \chi_i(w)^2 \lesssim \norm{L^\infty(D)}{w}^2 \quad \forall w \in C(\bar{D}) \label{eq:stab.p2},
% \end{align}
% where the hidden constants depend on $\Theta$ and $k$.

\subsection{Proof of Lemma \ref{lem:est.mesh.dependent}}\label{sec:pf:lem:est.mesh.dependent}
Observing that $\Pi^1_{k,h}((I - \Pi^1_{k,h})\Ih v) = 0$ due to linearity and idempotency of $\Pi^0_{k,h}$ we deduce that
\begin{align*}
    \norm{h_0}{\Ih{v} - \Pi^1_{k,h} \Ih{v}}^2 &= \sum_{D \in \Th} S^D((I - \Pi^1_{k,h}) \Ih v, (I - \Pi^1_{k,h}) \Ih v) \\
    &= \sum_{D \in \Th} \sum_{i=1}^{N_{\partial D}^{\texttt{dof}}} \chi_i((I - \Pi^1_{k,h}) \Ih v)^2 \\
    &\hspace{-0.15cm}\overset{\eqref{eq:stab.p2}}{\lesssim} \sum_{D \in \Th} \norm{L^\infty(D)}{(I - \Pi^1_{k,h}) \Ih v}^2 \\
    &\overset{\eqref{eq:Sobolev.inequality}}{\lesssim} \sum_{D \in \Th} h_D^{-2} \norm{L^2(D)}{(I - \Pi^1_{k,D}) \Ih v}^2 + \seminorm{H^1(D)}{(I - \Pi^1_{k,D}) \Ih v}^2 \\
    &\qquad+ h_D^{2\delta} \seminorm{H^{1+\delta}(D)}{(I - \Pi^1_{k,D}) \Ih v}^2 
\end{align*}
Now splitting $(I-\Pi^1_{k,D}) \Ih{v} = \Ih{v} - v + v - \Pi^1_{k,D} \Ih{v}$ and using the approximation properties of $\Ih$ from \eqref{eq:ID.approximation} and \eqref{eq:ID.approximation.L2}, we get the bound on the first term stated in \eqref{eq:est.Ih.h.norm}. Similarly, the second term in \eqref{eq:est.Ih.h.norm} follows from the definition of $\seminorm{1,h}{\cdot}$, the above mentioned split followed by a triangle inequality and the approximation properties of the interpolation operator stated in Lemma \ref{lem:ID.properties}.

\smallskip

Using the linearity and idempotency of $\Pi^1_{k,h}$ and \eqref{eq:stab.p2}, we can get the following
\begin{align*}
    \norm{h_\beta}{v- \Pi^1_{k,h}{v}}^2 &\lesssim \sum_{D \in \Th} \norm{L^\infty(D)}{v - \Pi^1_{k,D} v}^2 + \beta \norm{L^2(D)}{v - \Pi^1_{k,D} v}^2 \\
    &\hspace{-0.5cm}\overset{\eqref{eq:stab.p2},\eqref{eq:Sobolev.inequality}}{\lesssim} \sum_{D \in \Th} h_D^{-2} \norm{L^2(D)}{(I - \Pi^1_{k,D}) v}^2 + \seminorm{H^1(D)}{(I - \Pi^1_{k,D}) v}^2 \\
    &\quad\qquad+ h_D^{2\delta} \seminorm{H^{1+\delta}(D)}{(I - \Pi^1_{k,D}) v}^2 + \beta \norm{L^2(D)}{v - \Pi^1_{k,D} v}^2.
\end{align*}
Now using the approximation property of $\Pi^1_{k,D}$ in Lemma \ref{lem:H1p.properties} leads to the desired estimate on the first term in \eqref{eq:est.hbeta.norm}. Similarly, we have
\begin{align*}
    \norm{h_\beta}{v - \Ih{v}}^2 &= \sum_{D \in \Th} \seminorm{H^1(D)}{\Pi^1_{k,D}(v - \ID v)}^2 + \sum_{i=1}^{N_{\partial D}^{\texttt{dof}}} \chi_i((I-\Pi^1_{k,D})(I-\ID{})v)^2
    + \beta \norm{L^2(D)}{\Pi^0_{k,D}(v - \ID v)}^2 \\
    &= \sum_{D \in \Th} \seminorm{H^1(D)}{\Pi^1_{k,D}(v - \ID v)}^2 + \sum_{i=1}^{N_{\partial D}^{\texttt{dof}}} \chi_i(\Pi^1_{k,D}(I-\ID{})v)^2
    + \beta \norm{L^2(D)}{\Pi^0_{k,D}(v - \ID v)}^2 \\
    &\lesssim \sum_{D \in \Th} \seminorm{H^1(D)}{v - \ID v}^2 +  \norm{L^\infty(D)}{\Pi^1_{k,D}(I-\Pi^1_{k,D})v}^2 
    + \beta \norm{L^2(D)}{v - \ID v}^2,
\end{align*}    
where in the passage to the second equality we use the fact that $\chi_i(v - \ID{v}) = 0$ for boundary degrees of freedom by definition of interpolation operator \eqref{eq:ID.boundary}. In the last inequality we used the stability of projectors \eqref{eq:stability.H1p}, \eqref{eq:stability.L2p} and the property \eqref{eq:stab.p2}. The first and last terms are bounded using the approximation properties of $\ID$ from Lemma \ref{lem:ID.properties}. The second term can be estimated as in \cite[(4.24)]{brenner.sung:2018:virtual}. Combining all the resulting bounds leads to the desired estimate on the second term in \eqref{eq:est.hbeta.norm}.


























% \subsection{Proof of Lemma \ref{lem:Jh.approx}}\label{sec:pf:lem.Jh.approx}
%  We split the defect $ J_h v_h - v_h$ using $\Pi^0_{k,h}$ as 
%     \begin{align}
%         \norm{L^2(\Omega)}{J_h v_h - v_h} \leq \norm{L^2(\Omega)}{J_h v_h - \Pi^0_{k,h} v_h} + \norm{L^2(\Omega)}{\Pi^0_{k,h} v_h - v_h}. \label{eq:Jh.approximation.1}
%     \end{align}
%     The first term in \eqref{eq:Jh.approximation.1} is bounded using the definition of $J_h$ in \eqref{def:Jh} as follows:
%     \begin{align*}
%         \norm{L^2(\Omega)}{\Pi^0_{k,h} v_h - J_h v_h}^2 &= (\Pi^0_{k,h} v_h - J_h v_h,\Pi^0_{k,h} v_h - J_h v_h) \\
%         &= (\Pi^0_{k,h} v_h - J_h v_h, \Pi^0_{k,h} v_h) \\
%         &= (\Pi^0_{k,h} v_h  - J_h v_h, \Pi^0_{k,h} v_h - v_h).
%     \end{align*}
%     Thus, upon using Causchy-Schwarz inequality and \eqref{eq:approxL2.L2p}, we have
%     $$ \norm{L^2(\Omega)}{\Pi^0_{k,h} v_h - J_h v_h} \lesssim \left(\sum_{D \in \Th} \norm{L^2(D)}{ \Pi^0_{k,D} v_h - v_h}^2\right)^{1/2} \lesssim \left( \sum_{D \in \Th} h_D^2 \seminorm{H^1(D)}{v_h}^2 \right)^{1/2}.$$


% \subsection{Proof of Lemma \ref{lem:Jh.stability}}\label{sec:pf:lem.Jh.stability}
%    Testing with $v_h = J_h w$ and using Cauchy-Schwarz yields the stability estimate for $J_h$,
%     $$\norm{L^2(\Omega)}{J_h w} \leq \norm{L^2(\Omega)}{\Pi^0_{k,h} w} \overset{\eqref{eq:stability.L2p}}{\lesssim} \norm{L^2(\Omega)}{w}.$$

%     To prove the $H^1$-stability of $J_h$, we proceed as follows,
%     \begin{align}
%         \seminorm{H^1(\Omega)}{J_h v_h}^2 &\lesssim \seminorm{H^1(\Omega)}{J_h v_h - v_h}^2 + \seminorm{H^1(\Omega)}{v_h}^2
%         = \sum_{D \in \Th} \seminorm{H^1(D)}{J_h v_h - v_h}^2 + \seminorm{H^1(\Omega)}{v_h}^2. \label{eq:Jh.approxH1.1}
%     \end{align}
%     Using the inverse estimate in Lemma \ref{lem:inverse.inequality} for $\delta_h := J_h v_h - v_h \in V_h(D)$, we have
%     \begin{align}
%         \seminorm{H^1(D)}{\delta_h}^2 &\lesssim h_D^{-2} \tnorm{k,D}{\delta_h}^2 \nonumber\\
%         &= h_D^{-1} \sum_{e \in \ED} \norm{L^2(e)}{\Pi^0_{k,e} \delta_h}^2 + h_D^{-2}\norm{L^2(D)}{\Pi^0_{k-2,D} \delta_h}^2 \nonumber \\
%         &\leq h_D^{-1} \norm{L^2(\partial D)}{\delta_h}^2 + h_D^{-2}\norm{L^2(D)}{\delta_h}^2, \label{eq:Jh.approxH1.2}
%     \end{align}
%     where we used the stability of $\Pi^0_{k,e}$ and $\Pi^0_{k-2,D}$ from \eqref{eq:stability.L2p}. Now using a multiplicative trace inequality \cite[Theorem 1.6.6]{Brenner.Scott:2008:MTFEM} followed by Young's inequality for $\epsilon > 0$, yields
%     \begin{align*}
%         h_D^{-1} \norm{L^2(\partial D)}{\delta_h}^2 \lesssim h_D^{-1} \norm{L^2(D)}{\delta_h} \norm{H^1(D)}{\delta_h}
%         \leq \frac{1}{2\epsilon} h_D^{-2} \norm{L^2(D)}{\delta_h}^2 + \frac{\epsilon}{2} \norm{H^1(D)}{\delta_h}^2.
%     \end{align*}
%     Substituting the above in \eqref{eq:Jh.approxH1.2}, gives
%     \begin{align*}
%         \seminorm{H^1(D)}{\delta_h}^2 \lesssim \frac{1}{2\epsilon} h_D^{-2} \norm{L^2(D)}{\delta_h}^2 + \frac{\epsilon}{2} \norm{H^1(D)}{\delta_h}^2.
%     \end{align*}
%     Hence, applying the approximation property of $J_h$ from Lemma \ref{lem:Jh.approx} {\color{blue} alongwith quasi-uniformity of the mesh}, and Poincar\'e-Friedrichs inequality yields    
%     \begin{align*}
%         \seminorm{H^1(\Omega)}{\delta_h}^2 = \sum_{D \in \Th} \seminorm{H^1(D)}{\delta_h}^2 &\lesssim \left(1+\frac{1}{2\epsilon}\right) \sum_{D \in \Th} h_D^{-2} \norm{L^2(D)}{\delta_h}^2 + \frac{\epsilon}{2} \sum_{D \in \Th} \norm{H^1(D)}{\delta_h}^2 \nonumber\\
%         &\approx \left(1+\frac{1}{2\epsilon}\right) h^{-2} \sum_{D \in \Th}  \norm{L^2(D)}{\delta_h}^2 + \frac{\epsilon}{2} \sum_{D \in \Th} \norm{H^1(D)}{\delta_h}^2 \nonumber\\
%         &\lesssim \left(1+\frac{1}{2\epsilon}\right) \sum_{D \in \Th} \seminorm{H^1(D)}{v_h}^2 + \frac{\epsilon}{2} \norm{H^1(\Omega)}{\delta_h}^2 \nonumber\\
%         &\lesssim\left(1+\frac{1}{2\epsilon}\right) \seminorm{H^1(\Omega)}{v_h}^2 + \frac{\epsilon}{2} \seminorm{H^1(\Omega)}{\delta_h}^2.
%     \end{align*}
%     Choose $\epsilon > 0$ sufficiently small to absorb the last term on the right-hand side into the left-hand side, and substitute the resulting bound in \eqref{eq:Jh.approxH1.1} to complete the proof. 

	% Application to discontinuous skeletal methods
	\section{Numerical Results}\label{sec:numexp}

We choose the boundary version of the classical \texttt{Dofi-Dofi} definition of the stabilization term (see \cite[Section 4.2]{brenner.sung:2018:virtual} and \cite{beirao-da-veiga.brezzi.ea:2013:basic}),
\begin{align}
    S^D(w_h, v_h) = \sum_{i=1}^{N_{\partial D}^{\texttt{dof}}} \chi_i(w_h) \; \chi_i(v_h) \quad \forall w_h, v_h \in V_h(D),
\end{align}
where operator $\chi_i(\cdot)$ associates the function with its $i$-th degree of freedom, and $N_{\partial D}^{\texttt{dof}}$ is the number of boundary degrees of freedom associated with the element in $D \in \Th$.

\subsection{Smooth solution on a convex simply connected domain.}\label{exp:smooth_sol}
    We consider the problem \eqref{eq:P} with $\gamma = \beta = 0$ and $\Omega = (0,1)^2$. Given $\phi(x,y) = \sin^3(\pi x) \sin^3(\pi y)$, the manufactured solution is given by
    \begin{align*}
        \bs{u} = \curl{\phi}, \quad \mbox{with} \quad \bs{f} = - \curl{(\Delta (\curl{\bs{u}}))}.
    \end{align*}
    We solve \eqref{eq:P} using the scheme described in Section \ref{sec:discretization} for orders $k = 1,2,3,4$ on a sequence of unstructured Voronoi meshes. Since the virtual element solution is not known explicitly inside the element, we compare the manufactured solution with suitable projection of the discrete solution. The errors we measure are as follows:
    \begin{align*}
        e_{\bs{u}_h} := \norm{L^2(\Omega)}{ \bs{u} - \bs{u}_h} &= \sqrt{\sum_{D \in \Th} \norm{L^2(D)}{\bs{u} - \bs{u}_h}^2}, \\
        e_{\xi_h} := \seminorm{1,h}{\xi - \Pi^{1}_{k,h} \xi_h} &= \sqrt{\sum_{D \in \Th} \seminorm{H^1(D)}{\xi - \Pi^{1}_{k,D} \xi_h}^2}, \quad \mbox{with} \quad \xi = \curl{\bs{u}}.
    \end{align*}
    and are reported in Tables \ref{Table:smooth.structured} and \ref{Table:smooth.unstructured} for $k=1,2,3$ on a sequence of structured and unstructured voronoi meshes, respectively (see Figure \ref{Fig:Square_mesh}). We recall that in this experiment $\bs{u}_h$ is computed using \eqref{uh.simplyConnected}. 
    % The experimental rates of convergence are observed to be optimal, i.e., $\mathcal{O}(h^{k})$ for $e_{\bs{u}_h}$ and $\mathcal{O}(h^{k})$ for $e_{\xi_h}$.

\begin{longtable}{@{}|l|l|l|l|l|l|@{}}
  \caption{Experimental errors and orders of convergence for Experiment \ref{exp:smooth_sol} on structured voronoi meshes for orders $k = 1,2,3$.}
  \label{Table:smooth.structured}\\
  \hline
  $\qquad h$ & $\;\;\;N^{\texttt{Dofs}}$ & $\qquad e_{\bs{u}_h}$ & $\;\; \texttt{rate}$ & $\qquad e_{\xi_h}$ & $\;\; \texttt{rate}$ \\ \hline
  \multicolumn{6}{|c|}{$k = 1$} \\ \hline
  2.9814e-01 & 90    & 1.1627 & -     & 9.8919e+01 & -     \\
  1.4907e-01 & 280   & 5.4896e-01 & 1.0827 & 5.1407e+01 & 0.9443 \\
  7.4536e-02 & 960   & 2.4669e-01 & 1.1540 & 2.5100e+01 & 1.0343 \\
  3.7268e-02 & 3520  & 1.1807e-01 & 1.0631 & 1.2425e+01 & 1.0145 \\
  1.8634e-02 & 13440 & 5.8296e-02 & 1.0181 & 6.1970e+00 & 1.0036 \\
  9.3169e-03 & 52480 & 2.9054e-02 & 1.0047 & 3.0971e+00 & 1.0006 \\ \hline

  \multicolumn{6}{|c|}{ $k = 2$} \\ \hline
  2.9814e-01 & 251    & 2.7454e-01 & -     & 3.0855e+01 & -     \\
  1.4907e-01 & 801    & 7.1278e-02 & 1.9455 & 8.6421e+00 & 1.8361 \\
  7.4536e-02 & 2801   & 1.8366e-02 & 1.9564 & 2.2548e+00 & 1.9384 \\
  3.7268e-02 & 10401  & 4.6543e-03 & 1.9804 & 5.7300e-01 & 1.9764 \\
  1.8634e-02 & 40001  & 1.1703e-03 & 1.9916 & 1.4422e-01 & 1.9903 \\
  9.3169e-03 & 156801 & 2.9336e-04 & 1.9962 & 3.6163e-02 & 1.9957 \\ \hline

  \multicolumn{6}{|c|}{$k = 3$} \\ \hline
  2.9814e-01 & 448    & 6.3438e-02 & -     & 8.2035e+00 & -     \\
  1.4907e-01 & 1443   & 8.7780e-03 & 2.8534 & 1.1781e+00 & 2.7997 \\
  7.4536e-02 & 5083   & 1.1387e-03 & 2.9465 & 1.5452e-01 & 2.9306 \\
  3.7268e-02 & 18963  & 1.4425e-04 & 2.9807 & 1.9661e-02 & 2.9744 \\
  1.8634e-02 & 73123  & 1.8128e-05 & 2.9923 & 2.4755e-03 & 2.9895 \\
  9.3169e-03 & 287043 & 2.2713e-06 & 2.9966 & 3.1044e-04 & 2.9953 \\ \hline
\end{longtable}

\begin{longtable}{@{}|l|l|l|l|l|l|@{}}
  \caption{Experimental errors and orders of convergence for Experiment \ref{exp:smooth_sol} on unstructured Voronoi meshes for orders $k = 1,2,3$.}
  \label{Table:smooth.unstructured}\\
  \hline
  $\qquad h$ & $\;\;\;N^{\texttt{Dofs}}$ & $\qquad e_{\bs{u}_h}$ & $\;\; \texttt{rate}$ & $\qquad e_{\xi_h}$ & $\;\; \texttt{rate}$ \\ \hline
  \multicolumn{6}{|c|}{$k = 1$} \\ \hline
  3.7637e-01 & 48    & 1.1963 & -     & 1.0194e+02 & -     \\
  1.8360e-01 & 184   & 5.3872e-01 & 1.1114 & 5.1507e+01 & 0.9510 \\
  8.4283e-02 & 1234  & 1.8091e-01 & 1.4016 & 1.8902e+01 & 1.2876 \\
  4.1405e-02 & 5947  & 8.1748e-02 & 1.1176 & 8.6614e+00 & 1.0979 \\
  2.0988e-02 & 26477 & 3.8903e-02 & 1.0929 & 4.1493e+00 & 1.0831 \\
  1.0757e-02 & 96734 & 2.0202e-02 & 0.9804 & 2.1549e+00 & 0.9803 \\ \hline

  \multicolumn{6}{|c|}{$k = 2$} \\ \hline
  3.7637e-01 & 145    & 3.0531e-01 & -     & 3.4261e+01 & -     \\
  1.8360e-01 & 567    & 7.2772e-02 & 1.9977 & 8.8858e+00 & 1.8800 \\
  8.4283e-02 & 3867   & 1.1100e-02 & 2.4152 & 1.3484e+00 & 2.4218 \\
  4.1405e-02 & 18693  & 2.3647e-03 & 2.1756 & 2.8656e-01 & 2.1789 \\
  2.0988e-02 & 82953  & 5.2424e-04 & 2.2171 & 6.3400e-02 & 2.2201 \\
  1.0757e-02 & 303467 & 1.4262e-04 & 1.9477 & 1.7275e-02 & 1.9453 \\ \hline

  \multicolumn{6}{|c|}{$k = 3$} \\ \hline
  3.7637e-01 & 267    & 6.7643e-02 & -     & 9.2612e+00 & -     \\
  1.8360e-01 & 1050   & 9.0071e-03 & 2.8088 & 1.2075e+00 & 2.8381 \\
  8.4283e-02 & 7200   & 5.2892e-04 & 3.6412 & 7.2091e-02 & 3.6199 \\
  4.1405e-02 & 34839  & 5.0638e-05 & 3.3008 & 6.7755e-03 & 3.3269 \\
  2.0988e-02 & 154429 & 5.4932e-06 & 3.2690 & 7.3215e-04 & 3.2747 \\
  1.0757e-02 & 565200 & 7.6463e-07 & 2.9503 & 1.0195e-04 & 2.9497 \\ \hline
\end{longtable}


\begin{figure}[htbp]
\centering
\begin{subfigure}[b]{0.48\linewidth}
    \centering
    \includegraphics[width=1\linewidth]{StVoro_Sq.pdf}
    \caption{Structured voronoi mesh}
    \label{fig:StVoro_sq}
\end{subfigure}
\hfill
\begin{subfigure}[b]{0.48\linewidth}
    \centering
    \includegraphics[width=1\linewidth]{Voro_Sq.pdf}
    \caption{Unstructured voronoi mesh}
    \label{fig:Voro_sq}
\end{subfigure}
\caption{}
\label{Fig:Square_mesh}
\end{figure}

\subsection{Smooth data on a convex simply connected domain.}\label{exp:smoothdata}

We keep the same domain and discretization introduced in Experiment \ref{exp:smooth_sol}. We take $\beta = \gamma = 0$, and choose 
$$\bs{f} = ((x^2+1) \sin(x) + x y^3 +2, (y^2+1) \cos(x) + x^3 y^2 - 1).$$

Then we solve the sequence of discrete problems \eqref{eq:Ph.rhoh}-\eqref{uh.simplyConnected} for orders $k = 1,2$. Since the exact solution is unknown, we compute relative errors
    \begin{align*}
        \texttt{rel} \; e_{\bs{u}_h}^i := \frac{\norm{L^2(\Omega)}{ \bs{u}_{h}^i - \bs{u}_{h}^{i+1}}}{\norm{L^2(\Omega)}{\bs{u}_{h}^{i+1}}}, \quad \mbox{and} \quad \texttt{rel} \; e_{\xi_h}^i := \frac{\seminorm{1,h_{i+1}}{\Pi^{1,i}_{k,h} \xi_h^{i} - \Pi^{1,i+1}_{k,h} \xi_h^{i+1}}}{\seminorm{1,h_{i+1}}{\Pi^{1,i+1}_{k,h} \xi_h^{i+1}}},
    \end{align*}
    where $i$ denoetes the mesh level. Then we report them in Tables \ref{Table:smoothdata.structured} and \ref{Table:smoothdata.unstructured} on a sequence of structured and unstructured voronoi meshes, respectively (see Figures \ref{Fig:Square_mesh}).

\begin{longtable}{@{}|l|l|l|l|l|l|@{}}
  \caption{Relative errors and convergence rates for Experiment \ref{exp:smoothdata} on structured voronoi meshes for orders $k = 1,2$.}
  \label{Table:smoothdata.structured}\\
  \hline
  $\qquad h$ & $\;\;\;N^{\texttt{Dofs}}$ & $\quad \texttt{rel}\;e_{\bs{u}_h}$ & $\;\; \texttt{rate}$ & $\quad \texttt{rel}\;e_{\xi_h}$ & $\;\; \texttt{rate}$ \\ \hline
  \multicolumn{6}{|c|}{$k = 1$} \\ \hline
  2.9814e-01 & 90    & - & -      & - & -      \\
  1.4907e-01 & 280   & 2.3814e-01 & -      & 4.6021e-01 & -      \\
  7.4536e-02 & 960   & 1.2406e-01 & 0.9408 & 2.7185e-01 & 0.7595 \\
  3.7268e-02 & 3520  & 6.2627e-02 & 0.9862 & 1.4889e-01 & 0.8686 \\
  1.8634e-02 & 13440 & 3.1449e-02 & 0.9938 & 7.8161e-02 & 0.9297 \\
  9.3169e-03 & 52480 & 1.5765e-02 & 0.9963 & 4.0092e-02 & 0.9631 \\ \hline

  \multicolumn{6}{|c|}{$k = 2$} \\ \hline
  2.9814e-01 & 251    & - & -      & - & -      \\
  1.4907e-01 & 801    & 3.2060e-02 & -      & 8.4945e-02 & -      \\
  7.4536e-02 & 2801   & 1.0155e-02 & 1.6586 & 2.8971e-02 & 1.5519 \\
  3.7268e-02 & 10401  & 2.9260e-03 & 1.7952 & 9.0584e-03 & 1.6773 \\
  1.8634e-02 & 40001  & 7.8719e-04 & 1.8941 & 2.6825e-03 & 1.7557 \\
  9.3169e-03 & 156801 & 2.0419e-04 & 1.9468 & 7.6646e-04 & 1.8073 \\ \hline
\end{longtable}

\begin{longtable}{@{}|l|l|l|l|l|l|@{}}
  \caption{Relative errors and convergence rates for Experiment \ref{exp:smoothdata} on unstructured voronoi meshes for orders $k = 1,2$.}
  \label{Table:smoothdata.unstructured}\\
  \hline
  $\qquad h$ & $\;\;\;N^{\texttt{Dofs}}$ & $\quad \texttt{rel}\;e_{\bs{u}_h}$ & $\;\; \texttt{rate}$ & $\quad \texttt{rel}\;e_{\xi_h}$ & $\;\; \texttt{rate}$ \\ 
  \hline
  \multicolumn{6}{|c|}{$k = 1$} \\ \hline
  3.7637e-01 & 48    & - & -      & - & -      \\
  1.8360e-01 & 184   & 2.5848e-01 & -      & 4.9213e-01 & -      \\
  8.4283e-02 & 1234  & 1.2341e-01 & 0.9496 & 2.7482e-01 & 0.7483 \\
  4.1405e-02 & 5947  & 4.6503e-02 & 1.3731 & 1.1872e-01 & 1.1809 \\
  2.0988e-02 & 26477 & 2.1339e-02 & 1.1464 & 5.3990e-02 & 1.1597 \\
  1.0757e-02 & 96734 & 1.0318e-02 & 1.0873 & 2.6389e-02 & 1.0710 \\ \hline

  \multicolumn{6}{|c|}{$k = 2$} \\ \hline
  3.7637e-01 & 145    & - & -      & - & -      \\
  1.8360e-01 & 567    & 3.8655e-02 & -      & 1.1907e-01 & -      \\
  8.4283e-02 & 3867   & 1.1807e-02 & 1.5233 & 3.7556e-02 & 1.4821 \\
  4.1405e-02 & 18693  & 2.0839e-03 & 2.4402 & 7.5986e-03 & 2.2481 \\
  2.0988e-02 & 82953  & 4.1737e-04 & 2.3666 & 1.9587e-03 & 1.9952 \\
  1.0757e-02 & 303467 & 9.6109e-05 & 2.1971 & 3.8104e-04 & 2.4494 \\ \hline
\end{longtable}


\subsection{Piecewise smooth data on a non-convex simply connected domain.}\label{exp:nonsmoothdata}

We solve \eqref{eq:P} on a L-shaped domain $\Omega = (-1,1)^2 \; \backslash \; [0,1) \times (-1,0]$ with $\beta = \gamma = 0$, and a piecewise constant vector field $\bs{f}$,
$$\bs{f} = \begin{cases}
    \left[ 1/4, 5/4 \right]^\top, \quad \mbox{if} \quad |x| < 2^{-1/2}, \\
    \left[ 1/2, 3/2 \right]^\top, \quad \mbox{if} \quad 2^{-1/2} \leq |x| < 1, \\
    \left[ 1, 2 \right]^\top, \;\;\;\qquad \mbox{if} \quad |x| \geq 1.
\end{cases}$$
The relative errors are reported in Table \ref{Table:nonsmoothdata.structured} and \ref{Table:nonsmoothdata.unstructured} on a sequence of structured and unstructured voronoi meshes (see Figure \ref{Fig:Lshape_mesh}) for orders $k = 1,2$.


\begin{figure}[htbp]
\centering
\begin{subfigure}[b]{0.48\linewidth}
    \centering
    \includegraphics[width=1\linewidth]{StVoro_L.pdf}
    \caption{Structured voronoi mesh}
    \label{fig:stVoro_L}
\end{subfigure}
\hfill
\begin{subfigure}[b]{0.48\linewidth}
    \centering
    \includegraphics[width=1\linewidth]{Voro_L.pdf}
    \caption{Unstructured voronoi mesh}
    \label{fig:Voro_L}
\end{subfigure}
\caption{}
\label{Fig:Lshape_mesh}
\end{figure}

\begin{longtable}{@{}|l|l|l|l|l|l|@{}}
  \caption{Relative errors and convergence rates for Experiment \ref{exp:nonsmoothdata} on structured voronoi meshes for orders $k = 1,2$.}
  \label{Table:nonsmoothdata.structured}\\
  \hline
  $\qquad h$ & $\;\;\;N^{\texttt{Dofs}}$ & $\quad \texttt{rel}\;e_{\bs{u}_h}$ & $\;\; \texttt{rate}$ & $\quad \texttt{rel}\;e_{\xi_h}$ & $\;\; \texttt{rate}$ \\ \hline
  \multicolumn{6}{|c|}{$k = 1$} \\ \hline
  2.9814e-01 & 230    & - & -      & - & -      \\
  1.4907e-01 & 760    & 1.7658e-01 & -      & 3.1600e-01 & -      \\
  7.4536e-02 & 2720   & 1.0229e-01 & 0.7876 & 1.7874e-01 & 0.8221 \\
  3.7268e-02 & 10240  & 6.0474e-02 & 0.7583 & 9.5571e-02 & 0.9032 \\
  1.8634e-02 & 39680  & 3.6354e-02 & 0.7342 & 4.9551e-02 & 0.9476 \\
  9.3169e-03 & 156160 & 2.2161e-02 & 0.7141 & 2.5265e-02 & 0.9718 \\ \hline

  \multicolumn{6}{|c|}{$k = 2$} \\ \hline
  2.9814e-01 & 651    & - & -      & - & -      \\
  1.4907e-01 & 2201   & 7.7042e-02 & -      & 4.6187e-02 & -      \\
  7.4536e-02 & 8001   & 4.8020e-02 & 0.6820 & 1.5768e-02 & 1.5505 \\
  3.7268e-02 & 30401  & 3.0151e-02 & 0.6714 & 5.4008e-03 & 1.5458 \\
  1.8634e-02 & 118401 & 1.8975e-02 & 0.6681 & 2.1096e-03 & 1.3562 \\
  9.3169e-03 & 467201 & 1.1949e-02 & 0.6672 & 1.0279e-03 & 1.0372 \\ \hline
\end{longtable}


\begin{longtable}{@{}|l|l|l|l|l|l|@{}}
  \caption{Relative errors and convergence rates for Experiment \ref{exp:nonsmoothdata} on unstructured voronoi meshes for orders $k = 1,2$.}
  \label{Table:nonsmoothdata.unstructured}\\
  \hline
  $\qquad h$ & $\;\;\;N^{\texttt{Dofs}}$ & $\quad \texttt{rel}\;e_{\bs{u}_h}$ & $\;\; \texttt{rate}$ & $\quad \texttt{rel}\;e_{\xi_h}$ & $\;\; \texttt{rate}$ \\ \hline
  \multicolumn{6}{|c|}{$k = 1$} \\ \hline
  6.4474e-01 & 47     & - & -      & - & -      \\
  3.2497e-01 & 174    & 3.1455e-01 & -      & 5.5756e-01 & -      \\
  1.5600e-01 & 783    & 1.4861e-01 & 1.0218 & 2.9441e-01 & 0.8702 \\
  7.6972e-02 & 4279   & 7.6425e-02 & 0.9414 & 1.4527e-01 & 0.9999 \\
  3.8974e-02 & 16213  & 4.1667e-02 & 0.8913 & 6.6347e-02 & 1.1516 \\
  1.9933e-02 & 68184  & 2.4315e-02 & 0.8033 & 3.3254e-02 & 1.0301 \\
  1.0296e-02 & 255458 & 1.6116e-02 & 0.6226 & 1.6667e-02 & 1.0457 \\ \hline

  \multicolumn{6}{|c|}{$k = 2$} \\ \hline
  6.4474e-01 & 143    & - & -      & - & -      \\
  3.2497e-01 & 547    & 8.2224e-02 & -      & 1.5327e-01 & -      \\
  1.5600e-01 & 2465   & 4.4343e-02 & 0.8414 & 4.9060e-02 & 1.5522 \\
  7.6972e-02 & 13557  & 2.7502e-02 & 0.6762 & 1.3178e-02 & 1.8607 \\
  3.8974e-02 & 51425  & 1.5440e-02 & 0.8483 & 3.0375e-03 & 2.1564 \\
  1.9933e-02 & 216367 & 9.2966e-03 & 0.7565 & 1.1150e-03 & 1.4946 \\ 
  1.0296e-02 & 810915 & 6.4954e-03 & 0.5428 & 6.6147e-04 & 0.7904 \\\hline
\end{longtable}


\subsection{Piecewise smooth data on a multiply connected domain with Betti number 1.}\label{exp:Betti1}

In this experiment we consider the domain $\Omega = (0,1)^2 \; \backslash \; [1/4, 3/4]^2$ with Betti number $1$. We choose $\beta = \gamma = 1$ and the same piecewise smooth data $\bs{f}$ used in Experiment \ref{exp:nonsmoothdata}. We solve the sequence of discrete problems \eqref{eq:Ph.zeta.chi}-\eqref{eq:Ph.varphi.c} and post-process $\bs{u}_h$ using the relation \eqref{uh.notSimplyConnected} for orders $k = 1,2$. The relative errors are reported in Table \ref{Table:Betti1.struc} and \ref{Table:Betti1.unstruc} on a sequence of structured and unstructured voronoi meshes, respectively (see Figure \ref{Fig:Betti1_mesh}). We also report the relative error in the coefficient $c_j$,
$$\texttt{rel} \; e_{c_j}^i = \frac{|c_j^i - c_j^{i+1}|}{|c_j^{i+1}|},$$
as well as the values of $c_j$ for orders $k = 1,2$


\begin{figure}[htbp]
\centering
\begin{subfigure}[b]{0.48\linewidth}
    \centering
    \includegraphics[width=1\linewidth]{StVoro_Betti1.pdf}
    \caption{Structured voronoi mesh}
    \label{fig:stVoro_Betti1}
\end{subfigure}
\hfill
\begin{subfigure}[b]{0.48\linewidth}
    \centering
    \includegraphics[width=1\linewidth]{Voro_Betti1.pdf}
    \caption{Unstructured voronoi mesh}
    \label{fig:Voro_Betti1}
\end{subfigure}
\caption{}
\label{Fig:Betti1_mesh}
\end{figure}



\begin{longtable}{@{}|l|l|l|l|l|l|l|l|l|l|@{}}
  \caption{Relative errors and convergence rates for Experiment \ref{exp:Betti1} for orders $k = 1,2$ on structured voronoi meshes.}
  \label{Table:Betti1.struc}\\
  \hline
  $\quad h$ & $N^{\texttt{Dofs}}$ &
  $\texttt{rel}\;e_{\bs{u}_h}$ & $\texttt{rate}$ &
  $\texttt{rel}\;e_{\xi_h}$ & $\texttt{rate}$ &
  $\quad c_1$ & $\texttt{rate}$ &
  $\texttt{rel}\;e_{c_1}$ & $\texttt{rate}$ \\ \hline

  \multicolumn{10}{|c|}{$k = 1$} \\ \hline
  2.64e-01 & 96     & -        & -    & -        & -    & -0.15092 & -    & -        & -    \\
  1.67e-01 & 200    & 5.55e-01 & -    & 5.56e-01 & -    & -0.15258 & -    & 1.09e-02 & -    \\
  8.85e-02 & 644    & 3.96e-01 & 0.53 & 4.78e-01 & 0.24 & -0.15201 & 1.68 & 3.74e-03 & 1.68 \\
  4.77e-02 & 1930   & 2.36e-01 & 0.84 & 2.65e-01 & 0.95 & -0.15187 & 2.20 & 9.57e-04 & 2.20 \\
  2.40e-02 & 6584   & 1.48e-01 & 0.68 & 1.55e-01 & 0.78 & -0.15180 & 1.08 & 4.57e-04 & 1.08 \\
  1.18e-02 & 24092  & 9.62e-02 & 0.61 & 8.80e-02 & 0.80 & -0.15177 & 1.20 & 1.96e-04 & 1.20 \\
  6.11e-03 & 93334  & 6.09e-02 & 0.69 & 4.97e-02 & 0.86 & -0.15175 & 1.24 & 8.66e-05 & 1.24 \\ \hline

  \multicolumn{10}{|c|}{$k = 2$} \\ \hline
  2.64e-01 & 264    & -        & -    & -        & -    & -0.15218 & -     & -        & -     \\
  1.67e-01 & 552    & 2.30e-01 & -    & 1.80e-01 & -    & -0.15214 & -     & 3.20e-04 & -     \\
  8.85e-02 & 1840   & 2.01e-01 & 0.22 & 1.33e-01 & 0.48 & -0.15193 & -2.25 & 1.33e-03 & -2.25 \\
  4.77e-02 & 5614   & 1.14e-01 & 0.91 & 6.44e-02 & 1.17 & -0.15182 & 0.89  & 7.66e-04 & 0.89  \\
  2.40e-02 & 19408  & 7.48e-02 & 0.62 & 3.88e-02 & 0.74 & -0.15178 & 1.65  & 2.47e-04 & 1.65  \\
  1.18e-02 & 71592  & 4.98e-02 & 0.58 & 2.56e-02 & 0.59 & -0.15176 & 0.85  & 1.35e-04 & 0.85  \\
  6.11e-03 & 278634 & 3.20e-02 & 0.67 & 1.63e-02 & 0.69 & -0.15175 & 1.41  & 5.30e-05 & 1.41  \\ \hline
\end{longtable}







\begin{longtable}{@{}|l|l|l|l|l|l|l|l|l|@{}}
  \caption{Relative errors and convergence rates for Experiment \ref{exp:Betti1} for orders $k = 1,2$ on unstructured voronoi meshes.}
  \label{Table:Betti1.unstruc}\\
  \hline
  $\quad h$ & $N^{\texttt{Dofs}}$ &
  $\texttt{rel}\;e_{\bs{u}_h}$ & $\texttt{rate}$ &
  $\texttt{rel}\;e_{\xi_h}$ & $\texttt{rate}$ &
  $\quad c_1$ & $\texttt{rel}\;e_{c_1}$ & $\texttt{rate}$ \\ \hline

  \multicolumn{9}{|c|}{$k = 1$} \\ \hline
  3.01e-01 & 66     & -        & -    & -        & -    & -0.14966 & -        & -    \\
  1.41e-01 & 197    & 5.59e-01 & -    & 6.38e-01 & -    & -0.15085 & 7.93e-03 & -    \\
  7.32e-02 & 908    & 2.99e-01 & 0.95 & 3.72e-01 & 0.83 & -0.15136 & 3.35e-03 & 1.32 \\
  3.41e-02 & 4637   & 1.72e-01 & 0.72 & 1.92e-01 & 0.86 & -0.15162 & 1.68e-03 & 0.90 \\
  1.57e-02 & 17139  & 9.62e-02 & 0.75 & 9.46e-02 & 0.92 & -0.15169 & 5.14e-04 & 1.53 \\
  7.55e-03 & 92388  & 6.22e-02 & 0.60 & 5.32e-02 & 0.79 & -0.15173 & 2.47e-04 & 1.00 \\ \hline

  \multicolumn{9}{|c|}{$k = 2$} \\ \hline
  3.01e-01 & 204     & -        & -    & -        & -    & -0.15137 & -        & -    \\
  1.41e-01 & 614     & 1.85e-01 & -    & 1.59e-01 & -    & -0.15157 & 1.33e-03 & -    \\
  7.32e-02 & 2816    & 1.16e-01 & 0.71 & 7.82e-02 & 1.08 & -0.15168 & 7.04e-04 & 0.97 \\
  3.41e-02 & 14274   & 6.81e-02 & 0.70 & 4.08e-02 & 0.85 & -0.15172 & 3.13e-04 & 1.06 \\
  1.57e-02 & 52278   & 3.88e-02 & 0.73 & 2.27e-02 & 0.76 & -0.15174 & 9.70e-05 & 1.51 \\
  7.55e-03 & 284780  & 2.61e-02 & 0.54 & 1.50e-02 & 0.56 & -0.15174 & 4.20e-05 & 1.14 \\ \hline
\end{longtable}



\subsection{Piecewise smooth data on a multiply connected domain with Betti number 2.}\label{exp:Betti2}

We solve \eqref{eq:P} on a domain $\Omega = (-1,1)^2 \; \backslash \; [1/4, 3/4]^2 \cup [-1/4, -3/4]^2$ with Betti number $2$. We choose $\beta = \gamma = 1$ and the same piecewise smooth data $\bs{f}$ used in Experiment \ref{exp:nonsmoothdata}. The relative errors are reported in Table \ref{Table:Betti2} on a sequence of structured voronoi meshes (see Figure \ref{Fig:Betti2_mesh}). We also report the relative error in the coefficients $c_1, c_2$ and their values.

\begin{figure}[htbp]
\centering
\begin{subfigure}[b]{0.48\linewidth}
    \centering
    \includegraphics[width=1\linewidth]{StVoro_Betti2.pdf}
    \caption{Structured voronoi mesh}
    \label{fig:stVoro_Betti2}
\end{subfigure}
\hfill
\begin{subfigure}[b]{0.48\linewidth}
    \centering
    \includegraphics[width=1\linewidth]{Voro_Betti2.pdf}
    \caption{Unstructured voronoi mesh}
    \label{fig:Voro_Betti2}
\end{subfigure}
\caption{}
\label{Fig:Betti2_mesh}
\end{figure}


\begin{longtable}{@{}|l|l|l|l|l|l|l|l|l|l|l|l|@{}}
  \caption{Relative errors and convergence rates for Experiment \ref{exp:Betti2} for orders $k = 1,2$ on structured meshes.}
  \label{Table:Betti2}\\
  \hline
  $\quad h$ & $N^{\texttt{Dofs}}$ &
  $\texttt{rel}\;e_{\bs{u}_h}$ & $\texttt{rate}$ &
  $\texttt{rel}\;e_{\xi_h}$ & $\texttt{rate}$ &
  $\quad c_1$ & $\texttt{rel}\;e_{c_1}$ & $\texttt{rate}$ &
  $\quad c_2$ & $\texttt{rel}\;e_{c_2}$ & $\texttt{rate}$ \\ \hline

  \multicolumn{12}{|c|}{$k = 1$} \\ \hline
  4.35e-01 & 178    & -        & -    & -        & -    & -0.29668 & -        & -     & -0.13608 & -        & -    \\
  2.20e-01 & 470    & 5.00e-01 & -    & 3.92e-01 & -    & -0.29731 & 2.11e-03 & -     & -0.13590 & 1.37e-03 & -    \\
  1.18e-01 & 1462   & 3.68e-01 & 0.49 & 2.51e-01 & 0.71 & -0.29774 & 1.46e-03 & 0.58  & -0.13602 & 9.36e-04 & 0.61 \\
  6.02e-02 & 5054   & 2.31e-01 & 0.70 & 1.44e-01 & 0.83 & -0.29763 & 3.79e-04 & 2.02  & -0.13586 & 1.18e-03 & -0.35 \\
  3.06e-02 & 18014  & 1.46e-01 & 0.68 & 8.13e-02 & 0.84 & -0.29752 & 3.87e-04 & -0.03 & -0.13579 & 5.05e-04 & 1.25 \\
  1.52e-02 & 67916  & 9.48e-02 & 0.62 & 4.53e-02 & 0.84 & -0.29746 & 1.89e-04 & 1.03  & -0.13577 & 1.99e-04 & 1.33 \\ \hline

  \multicolumn{12}{|c|}{$k = 2$} \\ \hline
  4.35e-01 & 493     & -        & -    & -        & -    & -0.29863 & -        & -     & -0.13632 & -        & -    \\
  2.20e-01 & 1337    & 2.14e-01 & -    & 9.76e-02 & -    & -0.29837 & 8.88e-04 & -     & -0.13615 & 1.27e-03 & -    \\
  1.18e-01 & 4245    & 1.86e-01 & 0.22 & 6.10e-02 & 0.75 & -0.29780 & 1.92e-03 & -1.23 & -0.13592 & 1.68e-03 & -0.44 \\
  6.02e-02 & 14885   & 1.16e-01 & 0.70 & 3.06e-02 & 1.03 & -0.29759 & 7.06e-04 & 1.50  & -0.13582 & 7.35e-04 & 1.23 \\
  3.06e-02 & 53497   & 7.39e-02 & 0.67 & 1.93e-02 & 0.68 & -0.29749 & 3.28e-04 & 1.13  & -0.13578 & 3.18e-04 & 1.24 \\
  1.52e-02 & 202667 & 4.85e-02 & 0.60 & 1.23e-02 & 0.65 & -0.29744 & 1.44e-04 & 1.18  & -0.13576 & 1.44e-04 & 1.14 \\ \hline
\end{longtable}




\begin{longtable}{@{}|l|l|l|l|l|l|l|l|l|l|l|l|@{}}
  \caption{Relative errors and convergence rates for Experiment \ref{exp:Betti2} for orders $k = 1,2$ on unstructured meshes.}
  \label{Table:Betti2_unstruc}\\
  \hline
  $\quad h$ & $N^{\texttt{Dofs}}$ &
  $\texttt{rel}\;e_{\bs{u}_h}$ & $\texttt{rate}$ &
  $\texttt{rel}\;e_{\xi_h}$ & $\texttt{rate}$ &
  $\quad c_1$ & $\texttt{rel}\;e_{c_1}$ & $\texttt{rate}$ &
  $\quad c_2$ & $\texttt{rel}\;e_{c_2}$ & $\texttt{rate}$ \\ \hline

  \multicolumn{12}{|c|}{$k = 1$} \\ \hline
  3.25e-01 & 181    & -        & -    & -        & -    & -0.13483 & -        & -     & -0.29259 & -        & -    \\
  1.70e-01 & 770    & 4.10e-01 & -    & 3.37e-01 & -    & -0.13510 & 2.00e-03 & -     & -0.29611 & 1.19e-02 & -    \\
  8.30e-02 & 3093   & 2.48e-01 & 0.70 & 1.76e-01 & 0.91 & -0.13549 & 2.89e-03 & -0.52 & -0.29676 & 2.18e-03 & 2.37 \\
  4.08e-02 & 13033  & 1.49e-01 & 0.72 & 9.15e-02 & 0.92 & -0.13562 & 9.33e-04 & 1.59  & -0.29713 & 1.25e-03 & 0.78 \\
  2.00e-02 & 58140  & 9.48e-02 & 0.63 & 4.94e-02 & 0.87 & -0.13570 & 6.31e-04 & 0.55  & -0.29733 & 6.73e-04 & 0.87 \\ \hline

  \multicolumn{12}{|c|}{$k = 2$} \\ \hline
  3.25e-01 & 563    & -        & -    & -        & -    & -0.13555 & -        & -     & -0.29716 & -        & -    \\
  1.70e-01 & 2341   & 1.70e-01 & -    & 6.45e-02 & -    & -0.13565 & 7.63e-04 & -     & -0.29720 & 1.40e-04 & -    \\
  8.30e-02 & 9387   & 1.03e-01 & 0.70 & 3.15e-02 & 1.00 & -0.13572 & 4.81e-04 & 0.64  & -0.29731 & 3.57e-04 & -1.31 \\
  4.08e-02 & 39567  & 5.97e-02 & 0.77 & 1.61e-02 & 0.95 & -0.13573 & 8.30e-05 & 2.47  & -0.29737 & 2.10e-04 & 0.75 \\
  2.00e-02 & 176280 & 3.92e-02 & 0.59 & 1.09e-02 & 0.54 & -0.13574 & 9.27e-05 & -0.16 & -0.29740 & 1.11e-04 & 0.89 \\ \hline
\end{longtable}






    % The projected discrete solution profiles, $\Pi^0_{k,h} \bs{u}_h$ and $\Pi^{0}_{k,h} \xi_h$ are shown in Figure \ref{fig:solProfiles}, respectively. 

    % \begin{figure}
    %     \centering
    %     \includegraphics[width=0.9\linewidth]{EOC_uh.pdf}
    %     \caption{Order of convergence plots of $\norm{L^2(\Omega)}{ \bs{u} - \Pi^{0}_{k,h} \bs{u}_h}$ for the $k=1,2,3,4$, $H^1$-conforming VEM}
    %     \label{Table:1}
    % \end{figure}

    % \begin{figure}
    %     \centering
    %     \includegraphics[width=0.9\linewidth]{EOC_xi.pdf}
    %     \caption{Order of convergence plots of $\seminorm{H^1(\Omega)}{\curl{\bs{u}} - \Pi^{\nabla}_{k,h} \xi_h}$ for the $k=1,2,3,4$, $H^1$-conforming VEM}
    %     \label{Table:2}
    % \end{figure}

    % \begin{figure}
    %     \centering
    %     \includegraphics[width=0.3\linewidth]{proj_uh.pdf}
    %     \caption{Projected discrete solution $\Pi^0_{k,h} \bs{u}_h$}
    %     \label{fig:uh}
    % \end{figure}

    % \begin{figure}
    %     \centering
    %     \includegraphics[width=1\linewidth]{proj_xih.pdf}
    %     \caption{Projected discrete solution $\Pi^0_{k,h} \xi_h$}
    %     \label{fig:xih}
    % \end{figure}

    % \begin{figure}[htbp]
    % \centering
    % \begin{subfigure}[b]{0.48\linewidth}
    %     \centering
    %     \includegraphics[width=0.6\linewidth]{proj_uh.pdf}
    %     \caption{Projected discrete solution $\Pi^0_{k,h} \bs{u}_h$}
    %     \label{fig:uh}
    % \end{subfigure}
    % \hfill
    % \begin{subfigure}[b]{0.48\linewidth}
    %     \centering
    %     \includegraphics[width=1.7\linewidth]{proj_xih.pdf}
    %     \caption{Projected discrete solution $\Pi^0_{k,h} \xi_h$}
    %     \label{fig:xih}
    % \end{subfigure}
    % \caption{Projected discrete solution profiles.}
    % \label{fig:solProfiles}
    % \end{figure}

    % \subsection{Betti number 1.} Consider the domain $\Omega = (0,1)^2 \;\backslash\; [1/4,3/4]^2$, resembling a unit square with a hole. We take $$\beta = \gamma = 1, \quad \mbox{and} \quad \bs{f} = ((x^2+1) \sin(x) + x y^3 +2, (y^2+1) \cos(x) + x^3 y^2 - 1).$$

	\section*{Acknowledgments}
	$\dots$

	\printbibliography

\end{document}
